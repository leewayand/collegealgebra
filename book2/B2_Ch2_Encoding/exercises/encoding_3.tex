\documentclass{ximera}

\input{../../../preamble.tex}



\outcome{Identify pairs in a relation.}


\begin{document}








\begin{definition}
The graph below represent function pairs for the function L.
\begin{image}
\begin{tikzpicture}
	\begin{axis}[
            domain=-2:4,
            width=5in,
            height=3in,
            grid=both,
            axis lines =middle, xlabel=$y$, ylabel=$L$,
            xtick={-4, -3, -2, -1, 1, 2, 3, 4},
            ytick={-4, -3, -2, -1, 1, 2, 3, 4},
            every axis y label/.style={at=(current axis.above origin),anchor=south},
            every axis x label/.style={at=(current axis.right of origin),anchor=west},
            clip=false,
            %axis on top,
          ]
          \addplot [textColor, very thin, domain=(-5:5)] {0}; % puts the axis back, axis on top clobbers our open holes
          \addplot [textColor, very thin] plot coordinates {(0,-3) (0,3)}; % puts the axis back, axis on top clobbers our open holes

          \addplot [very thick, penColor, domain=(-2:1), <-] {0.5(x+3)};
          \addplot [very thick, penColor, domain=(1:3.5), ->] {1.5*sin(x * 180)-1};
          
          \addplot[color=penColor,fill=penColor,only marks,mark=*] coordinates{(1,2)};  %% closed hole    
          %\addplot[color=penColor,fill=penColor,only marks,mark=*] coordinates{(-1,2.7)};  %% closed hole   
          %\addplot[color=penColor,fill=penColor,only marks,mark=*] coordinates{(1.5,1)};  %% closed hole        
          %\addplot[color=penColor,fill=penColor,only marks,mark=*] coordinates{(3,1)};  %% closed hole        
          \addplot[color=penColor,fill=background,only marks,mark=*] coordinates{(1,4)};  %% open hole
          \addplot[color=penColor,fill=background,only marks,mark=*] coordinates{(1,-1)};  %% open hole
          
          %\node at  (axis cs:-3.3,-2) {A};
          %\node at  (axis cs:-2.2,0.3) {B};
          %\node at  (axis cs:1.3,-2) {C};
          %\node at  (axis cs:3.2,1) {D};

        \end{axis}
\end{tikzpicture}
\end{image}

\end{definition}





\begin{exercise}
L(-4)   $\approx \answer[given,tolerance=0.1]{-1}$
\end{exercise}


\begin{exercise}
L(-3)   $\approx \answer[given,tolerance=0.1]{0}$
\end{exercise}


\begin{exercise}
L(1)   $\approx \answer[given,tolerance=0.1]{2}$
\end{exercise}


\begin{exercise}
L(2)   $\approx \answer[given,tolerance=0.1]{-1}$
\end{exercise}


\begin{exercise}
L(4)   $\approx \answer[given,tolerance=0.1]{-1}$
\end{exercise}



\begin{exercise}
Solve   $ L\bigg( \answer[given,tolerance=0.1]{-1} \bigg) = 2$
\end{exercise}







\begin{exercise}
According to the graph, the domain of L is best described by which set?
\begin{multipleChoice}
\choice{$[-2, 3.5)$}
\choice{$(-\infty, 1) \cup (1, \infty)$}
\choice[correct]{$(-\infy, \infty)$}
\choice{$[-2, 3.5]$}
\end{multipleChoice}
\end{exercise}

\begin{exercise}
According to the graph, the range of L is best described by which set?
\begin{multipleChoice}
\choice{$[-2.4, 0.5] \cup [1, 4)$}
\choice[correct]{$(-\infty, \infty)$}
\choice{$[-2.4, 0.5] \cup [1, 4) \cup \{ 2 \}$}
\choice{$(-\infty, 4))$}
\end{multipleChoice}
\end{exercise}



\end{document}

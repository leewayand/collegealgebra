\documentclass{ximera}

\input{../../../preamble.tex}



\outcome{Identify pairs in a relation.}


\begin{document}








\begin{definition}
The graph below represent function pairs for the function f.
\begin{image}
\begin{tikzpicture}
	\begin{axis}[
            domain=-2:4,
            width=5in,
            height=3in,
            grid=both,
            axis lines =middle, xlabel=$x$, ylabel=$f$,
            xtick={-4, -3, -2, -1, 1, 2, 3, 4},
            ytick={-4, -3, -2, -1, 1, 2, 3, 4},
            every axis y label/.style={at=(current axis.above origin),anchor=south},
            every axis x label/.style={at=(current axis.right of origin),anchor=west},
            clip=false,
            %axis on top,
          ]
          \addplot [textColor, very thin, domain=(-5:5)] {0}; % puts the axis back, axis on top clobbers our open holes
          \addplot [textColor, very thin] plot coordinates {(0,-3) (0,3)}; % puts the axis back, axis on top clobbers our open holes

          \addplot [very thick, penColor, domain=(-3.5:-1), <-] {0.75(x+3)};
          \addplot [very thick, penColor, domain=(1:3.5), ->] {1.5*sin(x * 90)+1};
          
          \addplot[color=penColor,fill=penColor,only marks,mark=*] coordinates{(-1,-2)};  %% closed hole    
          %\addplot[color=penColor,fill=penColor,only marks,mark=*] coordinates{(-1,2.7)};  %% closed hole   
          %\addplot[color=penColor,fill=penColor,only marks,mark=*] coordinates{(1.5,1)};  %% closed hole        
          %\addplot[color=penColor,fill=penColor,only marks,mark=*] coordinates{(3,1)};  %% closed hole        
          \addplot[color=penColor,fill=background,only marks,mark=*] coordinates{(1,2.5)};  %% open hole
          \addplot[color=penColor,fill=background,only marks,mark=*] coordinates{(-1,2)};  %% open hole
          
          \node at  (axis cs:-1.8,1.5) {A};
          \node at  (axis cs:1.5,1.5) {B};
          \node at  (axis cs:-3.2,0.3) {C};
          \node at  (axis cs:2.6,0.3) {D};

        \end{axis}
\end{tikzpicture}
\end{image}

\end{definition}





\begin{exercise}
f(-4)   $\approx \answer[given,tolerance=0.1]{-1}$
\end{exercise}


\begin{exercise}
f(-1)   $\approx \answer[given,tolerance=0.1]{-2}$
\end{exercise}


\begin{exercise}
f(0)   $\approx \answer[given,tolerance=0.1]{DNE}$
\end{exercise}


\begin{exercise}
f(2)   $\approx \answer[given,tolerance=0.1]{1}$
\end{exercise}


\begin{exercise}
f(4)   $\approx \answer[given,tolerance=0.1]{1}$
\end{exercise}



\begin{exercise}
If you were solving $f(x) = 1.5$, then around which areas would you be looking?
\begin{selectAll}
\choice[correct]{$A$}
\choice[correct]{$B$}
\choice{$C$}
\choice{$D$}
\end{selectAll}
\end{exercise}



\begin{exercise}
$f(-5) \leq f(3)$
\begin{multipleChoice}
\choice[correct]{True}
\choice{False}
\choice{$Cannot Determine$}
\end{multipleChoice}
\end{exercise}



\begin{exercise}
$f(-1) \leq f(3)$
\begin{multipleChoice}
\choice[correct]{True}
\choice{False}
\choice{$Cannot Determine$}
\end{multipleChoice}
\end{exercise}


\begin{exercise}
$f(-7) \cdot f(-1) \leq 0$
\begin{multipleChoice}
\choice{True}
\choice[correct]{False}
\choice{$Cannot Determine$}
\end{multipleChoice}
\end{exercise}




\begin{exercise}
According to the graph, the domain of f is best described by which set?
\begin{multipleChoice}
\choice{$(-3.5, -1) \cup (1, 3.5)$}
\choice[correct]{$(-\infty, -1] \cup (1, \infty)$}
\choice{$(-3.5, -1] \cup (1, 3.5)$}
\choice{$(-3.5, -1] \cup (1, \infty)$}
\end{multipleChoice}
\end{exercise}

\begin{exercise}
According to the graph, the range of f is best described by which set?
\begin{multipleChoice}
\choice{$\{ -2 \} \cup (-0.5, 2.5)$}
\choice{$(-\infty, \infty)$}
\choice[correct]{$(-\infty, 2.5)$}
\choice{$(-\infty, 2.5]$}
\end{multipleChoice}
\end{exercise}



\end{document}

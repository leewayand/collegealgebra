\documentclass{ximera}

%\usepackage{todonotes}

\newcommand{\todo}{}

\usepackage{esint} % for \oiint
\graphicspath{
  {./}
  {ximeraTutorial/}
}

\newcommand{\mooculus}{\textsf{\textbf{MOOC}\textnormal{\textsf{ULUS}}}}

\usepackage{tkz-euclide}
\tikzset{>=stealth} %% cool arrow head
\tikzset{shorten <>/.style={ shorten >=#1, shorten <=#1 } } %% allows shorter vectors

\usetikzlibrary{backgrounds} %% for boxes around graphs
\usetikzlibrary{shapes,positioning}  %% Clouds and stars
\usetikzlibrary{matrix} %% for matrix
\usepgfplotslibrary{polar} %% for polar plots
\usetkzobj{all}
\usepackage[makeroom]{cancel} %% for strike outs
%\usepackage{mathtools} %% for pretty underbrace % Breaks Ximera
\usepackage{multicol}
\usepackage{pgffor} %% required for integral for loops


%% http://tex.stackexchange.com/questions/66490/drawing-a-tikz-arc-specifying-the-center
%% Draws beach ball 
\tikzset{pics/carc/.style args={#1:#2:#3}{code={\draw[pic actions] (#1:#3) arc(#1:#2:#3);}}}



\usepackage{array}
\setlength{\extrarowheight}{+.1cm}   
\newdimen\digitwidth
\settowidth\digitwidth{9}
\def\divrule#1#2{
\noalign{\moveright#1\digitwidth
\vbox{\hrule width#2\digitwidth}}}





\newcommand{\RR}{\mathbb R}
\newcommand{\R}{\mathbb R}
\newcommand{\N}{\mathbb N}
\newcommand{\Z}{\mathbb Z}

\newcommand{\sagemath}{\textsf{SageMath}}


%\renewcommand{\d}{\,d\!}
\renewcommand{\d}{\mathop{}\!d}
\newcommand{\dd}[2][]{\frac{\d #1}{\d #2}}
\newcommand{\pp}[2][]{\frac{\partial #1}{\partial #2}}
\renewcommand{\l}{\ell}
\newcommand{\ddx}{\frac{d}{\d x}}

\newcommand{\zeroOverZero}{\ensuremath{\boldsymbol{\tfrac{0}{0}}}}
\newcommand{\inftyOverInfty}{\ensuremath{\boldsymbol{\tfrac{\infty}{\infty}}}}
\newcommand{\zeroOverInfty}{\ensuremath{\boldsymbol{\tfrac{0}{\infty}}}}
\newcommand{\zeroTimesInfty}{\ensuremath{\small\boldsymbol{0\cdot \infty}}}
\newcommand{\inftyMinusInfty}{\ensuremath{\small\boldsymbol{\infty - \infty}}}
\newcommand{\oneToInfty}{\ensuremath{\boldsymbol{1^\infty}}}
\newcommand{\zeroToZero}{\ensuremath{\boldsymbol{0^0}}}
\newcommand{\inftyToZero}{\ensuremath{\boldsymbol{\infty^0}}}



\newcommand{\numOverZero}{\ensuremath{\boldsymbol{\tfrac{\#}{0}}}}
\newcommand{\dfn}{\textbf}
%\newcommand{\unit}{\,\mathrm}
\newcommand{\unit}{\mathop{}\!\mathrm}
\newcommand{\eval}[1]{\bigg[ #1 \bigg]}
\newcommand{\seq}[1]{\left( #1 \right)}
\renewcommand{\epsilon}{\varepsilon}
\renewcommand{\phi}{\varphi}


\renewcommand{\iff}{\Leftrightarrow}

\DeclareMathOperator{\arccot}{arccot}
\DeclareMathOperator{\arcsec}{arcsec}
\DeclareMathOperator{\arccsc}{arccsc}
\DeclareMathOperator{\si}{Si}
\DeclareMathOperator{\proj}{\vec{proj}}
\DeclareMathOperator{\scal}{scal}
\DeclareMathOperator{\sign}{sign}


%% \newcommand{\tightoverset}[2]{% for arrow vec
%%   \mathop{#2}\limits^{\vbox to -.5ex{\kern-0.75ex\hbox{$#1$}\vss}}}
\newcommand{\arrowvec}{\overrightarrow}
%\renewcommand{\vec}[1]{\arrowvec{\mathbf{#1}}}
\renewcommand{\vec}{\mathbf}
\newcommand{\veci}{{\boldsymbol{\hat{\imath}}}}
\newcommand{\vecj}{{\boldsymbol{\hat{\jmath}}}}
\newcommand{\veck}{{\boldsymbol{\hat{k}}}}
\newcommand{\vecl}{\boldsymbol{\l}}
\newcommand{\uvec}[1]{\mathbf{\hat{#1}}}
\newcommand{\utan}{\mathbf{\hat{t}}}
\newcommand{\unormal}{\mathbf{\hat{n}}}
\newcommand{\ubinormal}{\mathbf{\hat{b}}}

\newcommand{\dotp}{\bullet}
\newcommand{\cross}{\boldsymbol\times}
\newcommand{\grad}{\boldsymbol\nabla}
\newcommand{\divergence}{\grad\dotp}
\newcommand{\curl}{\grad\cross}
%\DeclareMathOperator{\divergence}{divergence}
%\DeclareMathOperator{\curl}[1]{\grad\cross #1}
\newcommand{\lto}{\mathop{\longrightarrow\,}\limits}

\renewcommand{\bar}{\overline}

\colorlet{textColor}{black} 
\colorlet{background}{white}
\colorlet{penColor}{blue!50!black} % Color of a curve in a plot
\colorlet{penColor2}{red!50!black}% Color of a curve in a plot
\colorlet{penColor3}{red!50!blue} % Color of a curve in a plot
\colorlet{penColor4}{green!50!black} % Color of a curve in a plot
\colorlet{penColor5}{orange!80!black} % Color of a curve in a plot
\colorlet{penColor6}{yellow!70!black} % Color of a curve in a plot
\colorlet{fill1}{penColor!20} % Color of fill in a plot
\colorlet{fill2}{penColor2!20} % Color of fill in a plot
\colorlet{fillp}{fill1} % Color of positive area
\colorlet{filln}{penColor2!20} % Color of negative area
\colorlet{fill3}{penColor3!20} % Fill
\colorlet{fill4}{penColor4!20} % Fill
\colorlet{fill5}{penColor5!20} % Fill
\colorlet{gridColor}{gray!50} % Color of grid in a plot

\newcommand{\surfaceColor}{violet}
\newcommand{\surfaceColorTwo}{redyellow}
\newcommand{\sliceColor}{greenyellow}




\pgfmathdeclarefunction{gauss}{2}{% gives gaussian
  \pgfmathparse{1/(#2*sqrt(2*pi))*exp(-((x-#1)^2)/(2*#2^2))}%
}


%%%%%%%%%%%%%
%% Vectors
%%%%%%%%%%%%%

%% Simple horiz vectors
\renewcommand{\vector}[1]{\left\langle #1\right\rangle}


%% %% Complex Horiz Vectors with angle brackets
%% \makeatletter
%% \renewcommand{\vector}[2][ , ]{\left\langle%
%%   \def\nextitem{\def\nextitem{#1}}%
%%   \@for \el:=#2\do{\nextitem\el}\right\rangle%
%% }
%% \makeatother

%% %% Vertical Vectors
%% \def\vector#1{\begin{bmatrix}\vecListA#1,,\end{bmatrix}}
%% \def\vecListA#1,{\if,#1,\else #1\cr \expandafter \vecListA \fi}

%%%%%%%%%%%%%
%% End of vectors
%%%%%%%%%%%%%

%\newcommand{\fullwidth}{}
%\newcommand{\normalwidth}{}



%% makes a snazzy t-chart for evaluating functions
%\newenvironment{tchart}{\rowcolors{2}{}{background!90!textColor}\array}{\endarray}

%%This is to help with formatting on future title pages.
\newenvironment{sectionOutcomes}{}{} 



%% Flowchart stuff
%\tikzstyle{startstop} = [rectangle, rounded corners, minimum width=3cm, minimum height=1cm,text centered, draw=black]
%\tikzstyle{question} = [rectangle, minimum width=3cm, minimum height=1cm, text centered, draw=black]
%\tikzstyle{decision} = [trapezium, trapezium left angle=70, trapezium right angle=110, minimum width=3cm, minimum height=1cm, text centered, draw=black]
%\tikzstyle{question} = [rectangle, rounded corners, minimum width=3cm, minimum height=1cm,text centered, draw=black]
%\tikzstyle{process} = [rectangle, minimum width=3cm, minimum height=1cm, text centered, draw=black]
%\tikzstyle{decision} = [trapezium, trapezium left angle=70, trapezium right angle=110, minimum width=3cm, minimum height=1cm, text centered, draw=black]




\outcome{Identify pairs in a relation.}


\begin{document}








\begin{definition}
The graph below represent function pairs for the function f.
\begin{image}
\begin{tikzpicture}
	\begin{axis}[
            domain=-2:4,
            width=5in,
            height=3in,
            grid=both,
            axis lines =middle, xlabel=$x$, ylabel=$f$,
            xtick={-4, -3, -2, -1, 1, 2, 3, 4},
            ytick={-4, -3, -2, -1, 1, 2, 3, 4},
            every axis y label/.style={at=(current axis.above origin),anchor=south},
            every axis x label/.style={at=(current axis.right of origin),anchor=west},
            clip=false,
            %axis on top,
          ]
          \addplot [textColor, very thin, domain=(-5:5)] {0}; % puts the axis back, axis on top clobbers our open holes
          \addplot [textColor, very thin] plot coordinates {(0,-3) (0,3)}; % puts the axis back, axis on top clobbers our open holes

          \addplot [very thick, penColor, domain=(-3.5:-1), <-] {0.75(x+3)};
          \addplot [very thick, penColor, domain=(1:3.5), ->] {1.5*sin(x * 90)+1};
          
          \addplot[color=penColor,fill=penColor,only marks,mark=*] coordinates{(-1,-2)};  %% closed hole    
          %\addplot[color=penColor,fill=penColor,only marks,mark=*] coordinates{(-1,2.7)};  %% closed hole   
          %\addplot[color=penColor,fill=penColor,only marks,mark=*] coordinates{(1.5,1)};  %% closed hole        
          %\addplot[color=penColor,fill=penColor,only marks,mark=*] coordinates{(3,1)};  %% closed hole        
          \addplot[color=penColor,fill=background,only marks,mark=*] coordinates{(1,2.5)};  %% open hole
          \addplot[color=penColor,fill=background,only marks,mark=*] coordinates{(-1,2)};  %% open hole
          
          \node at  (axis cs:-1.8,1.5) {A};
          \node at  (axis cs:1.5,1.5) {B};
          \node at  (axis cs:-3.2,0.3) {C};
          \node at  (axis cs:2.6,0.3) {D};

        \end{axis}
\end{tikzpicture}
\end{image}

\end{definition}





\begin{exercise}
f(-4)   $\approx \answer[given,tolerance=0.1]{-1}$
\end{exercise}


\begin{exercise}
f(-1)   $\approx \answer[given,tolerance=0.1]{-2}$
\end{exercise}


\begin{exercise}
f(0)   $\approx \answer[given,tolerance=0.1]{DNE}$
\end{exercise}


\begin{exercise}
f(2)   $\approx \answer[given,tolerance=0.1]{1}$
\end{exercise}


\begin{exercise}
f(4)   $\approx \answer[given,tolerance=0.1]{1}$
\end{exercise}



\begin{exercise}
If you were solving $f(x) = 1.5$, then around which areas would you be looking?
\begin{selectAll}
\choice[correct]{$A$}
\choice[correct]{$B$}
\choice{$C$}
\choice{$D$}
\end{selectAll}
\end{exercise}



\begin{exercise}
$f(-5) \leq f(3)$
\begin{multipleChoice}
\choice[correct]{True}
\choice{False}
\choice{$Cannot Determine$}
\end{multipleChoice}
\end{exercise}



\begin{exercise}
$f(-1) \leq f(3)$
\begin{multipleChoice}
\choice[correct]{True}
\choice{False}
\choice{$Cannot Determine$}
\end{multipleChoice}
\end{exercise}


\begin{exercise}
$f(-7) \cdot f(-1) \leq 0$
\begin{multipleChoice}
\choice{True}
\choice[correct]{False}
\choice{$Cannot Determine$}
\end{multipleChoice}
\end{exercise}




\begin{exercise}
According to the graph, the domain of f is best described by which set?
\begin{multipleChoice}
\choice{$(-3.5, -1) \cup (1, 3.5)$}
\choice[correct]{$(-\infty, -1] \cup (1, \infty)$}
\choice{$(-3.5, -1] \cup (1, 3.5)$}
\choice{$(-3.5, -1] \cup (1, \infty)$}
\end{multipleChoice}
\end{exercise}

\begin{exercise}
According to the graph, the range of f is best described by which set?
\begin{multipleChoice}
\choice{$\{ -2 \} \cup (-0.5, 2.5)$}
\choice{$(-\infty, \infty)$}
\choice[correct]{$(-\infty, 2.5)$}
\choice{$(-\infty, 2.5]$}
\end{multipleChoice}
\end{exercise}



\end{document}

\documentclass{ximera}

\input{../../preamble.tex}

\title{String Functions}

\begin{document}

\begin{abstract}
%Stuff can go here later if we want!
\end{abstract}

\maketitle






Some functions are just too big to actually list the pairs. 50 flags wasn't too bad, but what about a function that connects every grain of sand on the Earth with its weight? A group of researchers speaking with NPR estimated that there were roughly seven quintillion, five hundred quadrillion grains on the Earth?

SandWeight is a fine function.  Its domain and range are straightforward and you can probably imagine what the pairs might look like.  No grain of sand has two weights, so SandWeight is a well-defined function.  The structure is ok, but the mathematics is going to be difficult.  

Assuming you could weigh (and record) a grain of sand in a second, it would take you 2378234399 centuries to describe all of the pairs in this function.  

When the number of pairs are manageable, then we might create a list or a table  But, many times we merely provide instructions on how to create any particular pair you are interested in. You create the pairs as you need them or merely contemplate the pairs without actually recording any. Rather than listing the pairs, we just supply \textbf{assembly instructions}.

\textbf{ShiftLeft}  \\
Let's build a big function.  The domain and range will both be all strings of lowercase letters of any length whether they form a word or not.

Examples are "w", "tree", "jk", and "hfheurtutersg". 

There are a lot of strings of letters, more than grains of sand on the Earth. So, we can?t actually list the domain or range. We'll just have to go with our description and check each pair we construct as we construct them. 

In place of listing all of the pairs, we will provide some just-in-time assembly instructions on how to build pairs. Apply the procedure as you need.



\begin{definition} 
\begin{itemize}
\item Both the domain and range of the ShiftLeft function are the set of all strings of lowercase letters. 
\item The ShiftLeft function pairs a string from the domain with a string from the range. To create the range partner of a given domain string, take the first/left letter of the domain string and move it to the end/right of the string.
\end{itemize}
\end{definition}
\quad \\

Is ShiftLeft a well-defined function?

If you select any string of letters from the domain, then you
\begin{enumerate}
\item can move its first letter to the end, so it is in a pair.
\item only get one resulting string from this procedure, so it is in exactly one pair.
\end{enumerate}
  
Yes!  It is a well-defined function.  Paranoia satisfied. We can continue.


\begin{example}
\begin{itemize}
\item $ShiftLeft(abcdefg) = bcdefga$
\item $ShiftLeft(qw) = wq$
\item $ShiftLeft(school) = chools$
\end{itemize}
\end{example}


\begin{question}
Evaluate $ShiftLeft(bjkythf)$
\begin{multipleChoice}
\choice{$bjkythf$}
\choice[correct]{$jkythfb$}
\choice{$fbjkyth$}
\choice{$bjkythfb$}
\end{multipleChoice}
\end{question}


\begin{question}
What is the value of $ShiftLeft$ \textbf{at} $nvbhfy$?
\begin{multipleChoice}
\choice{$nvbhfy$}
\choice{$ynvbhfy$}
\choice{$nvbhfyn$}
\choice[correct]{$vbhfyn$}
\end{multipleChoice}
\end{question}


\begin{question}
What does $ShiftLeft$ of $wwtyuw$ equal?
\begin{multipleChoice}
\choice[correct]{$wtyuww$}
\choice{$tyuwww$}
\choice{$wwwtyu$}
\choice{$wwtyuww$}
\end{multipleChoice}
\end{question}


\begin{dialogue}
\item[\textbf{QUESTION}] Is it possible that a string is paired with itself?

\item[\textbf{ANSWER}] Yes. ShiftLeft(www) = www.  
\end{dialogue}
The pair (www, www) is in the ShiftLeft function. The string "www" is paired with itself.
\quad \\







\begin{remark} COMMMUNICATION \\
Solve $ShiftLeft(string) = ploik$ 


Is string acting as the variable representing domain items or is it the actual string "string" from the domain? Using words for variables in our function notation was ok when the domain held names of kindergarteners and states, but what do you do when the domain is strings of letters?

You switch!  Let's use a giraffe for our variable.

Solve $ShiftLeft($\includegraphics{pics/giraffe.png}$) = ploik$

Hmmm, perhaps a little too tall.

How about a crocodile?

Solve $ShiftLeft($\includegraphics{pics/alligator.png}$) = ploik$

That works.  The solution to this equation is 
\includegraphics{pics/alligator.png}$ = kploi$

\end{remark}



\begin{question}
Solve $ShiftLeft($\includegraphics{pics/alligator.png}$) = tree$
\begin{multipleChoice}
\choice{$tree$}
\choice{$etree$}
\choice{$etre$}
\end{multipleChoice}
\begin{feedback}
You have to "undo" the left shift.
\end{feedback}
\end{question}




\begin{question}
Solve $ShiftLeft($\includegraphics{pics/alligator.png}$) = kndgty$
\begin{multipleChoice}
\choice{$ndgtyk$}
\choice{$ykndgty$}
\choice{$ykndgt$}
\end{multipleChoice}
\begin{feedback}
You have to "undo" the left shift.
\end{feedback}
\end{question}



\begin{question}
Solve $ShiftLeft($\includegraphics{pics/alligator.png}$) = a$
\begin{multipleChoice}
\choice{$a$}
\choice{$aa$}
\choice{no solution}
\end{multipleChoice}
\begin{feedback}
You cannot have more letters than you began with.
\end{feedback}
\end{question}




\end{document}

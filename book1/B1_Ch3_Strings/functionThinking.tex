\documentclass{ximera}

\input{../../preamble.tex}

\title{Function Thinking}

\begin{document}

\begin{abstract}
%Stuff can go here later if we want!
\end{abstract}

\maketitle






\begin{question}
Solve $ShiftLeft($\includegraphics{pics/alligator.png}$) = a$
\begin{multipleChoice}
\choice{$a$}
\choice{$aa$}
\choice{no solution}
\end{multipleChoice}
\begin{feedback}
You cannot have more letters than you began with.
\end{feedback}
\end{question}




\begin{exploration}
Solve $ShiftLeft( $ \includegraphics{pics/alligator.png}$ ) = $ \includegraphics{pics/alligator.png} $)$ \\
That is, which strings are paired with themselves in the ShiftLeft function?
\quad \\

\textbf{Palindromes} \\
Palindromes are strings that read the same backward as forward.  Examples are civic, radar, and level.
\end{exploration}




\begin{exploration}

New Function: \textbf{ShiftRight} \\
Domain and range of ShiftRight are also the collection of all letter strings. To get the domain partner of a string from the domain, move the last letter to the front.  In effect, the letters shift right. \\


\begin{dialogue}
\item[\textbf{QUESTION}] Is ShiftRight well-defined?
\item[\textbf{ANSWER}] Yes. If you start with a word or string and move the last letter to the front, then you only get one word. 
\end{dialogue}
\quad \\


\begin{question}
Evaluate $ShiftRight(werty)$
\begin{multipleChoice}
\choice[correct]{$ywert$}
\choice{$ertyw$}
\choice{$wertyw$}
\choice{$ywerty$}
\end{multipleChoice}
\end{question}



\begin{question}
Evaluate $ShiftRight(mmmnjhuytree)$
\begin{multipleChoice}
\choice{$mmnjhuytreem$}
\choice[correct]{$emmmnjhuytre$}
\choice{$emmmnjhuytree$}
\choice{$emmmnjhuytreee$}
\end{multipleChoice}
\end{question}





\begin{question}
Evaluate $ShiftRight(ppppp)$
\begin{multipleChoice}
\choice{$pppp$}
\choice{$pppppp$}
\choice[correct]{$ppppp$}
\choice{$p$}
\end{multipleChoice}
\end{question}




\begin{question}
Solve $ShiftLeft($\includegraphics{pics/alligator.png}$) = ghjuty$
\begin{multipleChoice}
\choice{$ghjuty$}
\choice{$yghjut$}
\choice{$ghjut$}
\choice[correct]{$hjutyg$}
\end{multipleChoice}
\end{question}




\end{exploration}
\quad \\





\begin{exploration}
New Function: \textbf{LeftRightShift}\\
The LeftRightShift function first shifts the letters to the left, then it shifts them to the right to obtain the range partner of a string from the domain.
\quad \\


\begin{question}
Evaluate $LeftRightShift(kijusd)$
\begin{multipleChoice}
\choice{$ijusk$}
\choice{$dkijus$}
\choice[correct]{$kijusd$}
\choice{$dkijusd$}
\end{multipleChoice}
\end{question}





\begin{question}
Evaluate $LeftRightShift(mkwtgh)$
\begin{multipleChoice}
\choice{$hmkwtg$}
\choice{$kwtghm$}
\choice[correct]{$mkwtgh$}
\choice{$hgtwkm$}
\end{multipleChoice}
\end{question}




\begin{question}
Solve $LeftRightShift($\includegraphics{pics/alligator.png}$) = ftwqxm$
\begin{multipleChoice}
\choice[correct]{$ftwqxm$}
\choice{$twqxmf$}
\choice{$mftwqx$}
\choice{$mxqwtf$}
\end{multipleChoice}
\end{question}





\begin{dialogue}
\item[\textbf{QUESTION}] Solve $LeftRightShift($\includegraphics{pics/alligator.png}$ ) = $ \includegraphics{pics/alligator.png}
\item[\textbf{ANSWER}] Every word or string satisfes this equation.  It you shift left and then shift right you always get back the original word.
\end{dialogue}

\end{exploration}






\begin{exploration}
New Function: \textbf{NumberOfConsonants}\\
The NumberofConsonants function or NoC for short, pairs a string with a whole number. The domain is all strings of lowercase letters and the codomain is all whole numbers. Each string is paired with the number of consonants in the string.
Does the range = the codomain for this function?
Pretend that S1 and S2 are two strings and S1+S2 is a new string made from chaining together S1 and S2 (concatenating).  Explain why NoC(S1+S2) [~] NoC(S1).
\quad \\

\begin{dialogue}
\item[\textbf{QUESTION}] Does the range = the codomain for this function?
\item[\textbf{ANSWER}] No. The domain has strings and the range has numbers.
\end{dialogue}

\quad \\


\begin{dialogue}
\item[\textbf{QUESTION}] Pretend that $S1$ and $S2$ are two strings and $S1+S2$ is a new string made from chaining together $S1$ and $S2$ (concatenating).  Explain why $NoC(S1+S2) \geq NoC(S1)$.
\item[\textbf{ANSWER}] If you glue together two strings, then all you can do is possibly add more consonants. You don not take any away.
\end{dialogue}

\end{exploration}
\quad \\



\begin{exploration}
New Function: \textbf{DoubleVowel} is defined as follows \\
\begin{itemize}
\item Domain is all alphabet strings.
\item Codomain is all natural numbers.
\item $DoubleVowel(string)$ = number of consonants in $string + 2 \cdot$ number of vowels in $string$.
\end{itemize}

\quad \\




\begin{question}
Evaluate  $DoubleVowel(barn) = \answer{5}$
\end{question}


\begin{question}
Evaluate  $DoubleVowel(lake) = \answer{6}$
\end{question}


\begin{question}
Evaluate  $DoubleVowel(bookkeeper) = \answer{15}$
\end{question}


\begin{dialogue}
\item[\textbf{QUESTION}] Determine one solution of $DoubleVowel(string) = 3$.
\item[\textbf{ANSWER}] One possible solution is for the variable $string$ to equal the word $at$.
\end{dialogue}

\quad \\

\begin{dialogue}
\item[\textbf{QUESTION}] Is there a natural number not in the range?
\item[\textbf{ANSWER}] No. If the string has $m$ consonants with $k$ vowels, then $DoubleVowel$ has the value $m + 2 \cdot k$. 
For any natural number, $N$, we can values for $m$ and $k$, such that $m + 2 \cdot k = N$.
\end{dialogue}







\end{exploration}









\end{document}

\documentclass{ximera}

\input{../../preamble.tex}

\title{Preview}

\begin{document}

\begin{abstract}
%Stuff can go here later if we want!
\end{abstract}

\maketitle



Congratulations!  

You have made it to the beginning. 


Eventually, we will map out what are called the \textbf{Elementary Functions}.  The Elementary Functions make up the basic toolbox for Calculus. We will go through all of their definitions, properties, and graphs. But, first, we need to figure out what are \textbf{functions}.

Functions are the subject of this whole course.  So, we don?t want to skip corners.  We want the full story.  

Not only that, but we are laying a foundation for Calculus. So, we need more than just a procedural knowledge.  We need a conceptual viewpoint of the inner workings of functions.  

\begin{itemize}
\item We?ll begin with some situations involving collections of familiar information.  
\item We?ll ask common questions about the relationships between these collections.
\item We?ll organize our thoughts to help us thinking through the questions.
\item We?ll formalize these thoughts into structures called relations and functions.
\end{itemize}



We don't know this yet, but relations are too complex for us, right now. We?ll leave those for later abstract analysis courses. On the other hand, functions have just enough structure to be helpful and that's it.  In fact, functions have so little structure, it is a wonder that they are useful at all.

There is one problem.  Your expectation.

You have expectations based on your previous math courses and those expectations don't align very well with our intentions in this course.  What happens is that students focus on the numbers and try to calculate immediately to get an "answer".  In the process, students tend to by-pass all of the structure that we want to investigate. 

This chapter is going to help everyone out.  There are no numbers.  

Our collections will hold other types of information besides numbers. This way nobody will be tempted to jump over the structure. 
 The plan is to superficially develop the entire structure of functions using non-numeric collections. No numbers. We will map out an entire foundation for this course with pictures and drawings and words. 

By using drawings, we will gain a conceptual idea of functions.  Once we have a mental map of functions, we?ll bring in numbers. We?ll bring them in slowly, so that the don?t interfere with our construction of functions.

But, all of that is for a later book.  First, things first.  Let?s find out why we want to study functions.






\end{document}

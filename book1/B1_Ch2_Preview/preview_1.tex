\documentclass{ximera}

\input{../../preamble.tex}

\title{Preview of Chapter 2}

\begin{document}

\begin{abstract}
%Stuff can go here later if we want!
\end{abstract}

\maketitle


What is College Algebra?
and
Why are you studying it?


College Algebra is the first college mathematics course in the STEM (Science, Technology, Engineering, and Mathematics) pathway. It serves as a transition course.

Numbers to Functions?In previous mathematics courses, students generally encounter a computational view of mathematics in which numbers are the objects under investigation. College Algebra introduces a transition from numbers as the main focus of study to functions as the main focus. This transition is vital for success in the STEM pathway.
Almost anyone entering college would agree that numbers are the point of mathematics.  Students have spent years calculating with numbers, tracking down numbers from equations, and transitioning these procedural skills to algebraic symbols, which represent numbers. Well, that was phase 1.  Phase 1 is now complete.  Phase 2 is about to begin.
The world is too complicated to describe with individual numbers. Even collections of numbers are not enough. We need to understand how collections of numbers relate to each other. These relations are the focus of College Algebra.
Our goal is to create collections of numbers derived from measurements and describe how the collections compare to each other. To do this, we?ll create a new tool called a function.
Functions are the main focus of the STEM pathway.  College Algebra and Trigonometry lay the foundation for the Elementary functions and then Calculus develops the analysis of functions.  Functions describe the relationship between collections of numbers and Calculus deepens and expands the analysis of this relationship.
There is more to the College Algebra transition.

Conceptual Viewpoint?STEM students also need to transition from a procedural viewpoint of mathematics to a conceptual view. In previous mathematics courses, students worked with individual numbers. Solutions consisted of isolated numbers. Working with functions certainly involves individual and isolated numbers. But, these numbers are usually marking change in function behavior. It is the behavior of functions that we want to map out. Rather than a goal of identifying individual numbers, we will use these numbers to partition the collections of measurements, compare the pieces, and categorize the connections. Our viewpoint needs to expand from items to relationships between collections of items.
Calculus will then expand this idea.  The collections of items will be collections of functions and then there will be new functions created that describe relationships between the collections of functions.
There is more to the College Algebra transition.













\end{document}

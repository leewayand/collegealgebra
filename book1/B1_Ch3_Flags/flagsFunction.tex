\documentclass{ximera}

\input{../../preamble.tex}

\title{Flags Function}

\begin{document}

\begin{abstract}
%Stuff can go here later if we want!
\end{abstract}

\maketitle




\textbf{U.S. State Flags Function} \\

There is a connection between states and flags.  Each state can be paired with its state flag.  This connect is the basis for a relation.
In this way, each state (domain item) is paired with exactly one flag (codomain item), a function - the StateFlags function.


\begin{definition} 
\begin{itemize}
\item The domain of the StateFlags function is the set of all fifty states.
\item The codomain of the StateFlags function is the set of all fifty flags.
\item Each pair in the StateFlags function pairs a state with its state flag.
\end{itemize}
\end{definition}
\quad \\

So, far we have a perfectly good relation. But, just because someone says we have a function, doesn?t necessarily mean we do.  A little paranoia is needed here. We need to verify that it is a function.

\begin{itemize}
\item Does every state have a state flag?  Yes.
\item Does any state have more than one state flag?  No.
\item Is every state in exactly one pair?  Yes.
\end{itemize}


We have a function! 

\textbf{Note:}  The range = codomain for this function.
\quad \\




\begin{center}
\textbf{StateFlags Function} \\
Pairs represented via a table. \\

\includegraphics{pics/allStateFlags_med.png}
\end{center}







\begin{center}
\textbf{StateFlags Function} \\
Pairs represented via a table.
\end{center}



\begin{center}

\begin{tabular}{|l|l|l|l|}
\hline
Alabama & \includegraphics{pics/StateFlags/Alabama.png} & Montana & \includegraphics{pics/StateFlags/Montana.png} \\\hline 
Alaska & \includegraphics{pics/StateFlags/Alaska.png} & Nebraska & \includegraphics{pics/StateFlags/Nebraska.png} \\\hline 
Arizona & \includegraphics{pics/StateFlags/Arizona.png} & Nevada & \includegraphics{pics/StateFlags/Nevada.png} \\\hline 
Arkansas & \includegraphics{pics/StateFlags/Arkansas.png} & New Hampshire & \includegraphics{pics/StateFlags/NewHampshire.png} \\\hline 
California & \includegraphics{pics/StateFlags/California.png} & New Jersey & \includegraphics{pics/StateFlags/NewJersey.png} \\\hline 
Colorado & \includegraphics{pics/StateFlags/Colorado.png} & New Mexico & \includegraphics{pics/StateFlags/NewMexico.png} \\\hline 
Connecticut & \includegraphics{pics/StateFlags/Connecticut.png} & NewYork & \includegraphics{pics/StateFlags/NewYork.png} \\\hline 
Delaware & \includegraphics{pics/StateFlags/Delaware.png} & North Carolina & \includegraphics{pics/StateFlags/NorthCarolina.png} \\\hline 
Florida & \includegraphics{pics/StateFlags/Florida.png} & North Dakota & \includegraphics{pics/StateFlags/NorthDakota.png} \\\hline 
Georgia & \includegraphics{pics/StateFlags/Georgia.png} & Ohio & \includegraphics{pics/StateFlags/Ohio.png} \\\hline 
Hawaii & \includegraphics{pics/StateFlags/Hawaii.png} & Oklahoma & \includegraphics{pics/StateFlags/Oklahoma.png} \\\hline 
Idaho & \includegraphics{pics/StateFlags/Idaho.png} & Oregon & \includegraphics{pics/StateFlags/Oregon.png} \\\hline 
\end{tabular}



\begin{tabular}{|l|l|l|l|}
\hline
Illinois & \includegraphics{pics/StateFlags/Illinois.png} & Pennsylvania & \includegraphics{pics/StateFlags/Pennsylvania.png} \\\hline 
Indiana & \includegraphics{pics/StateFlags/Indiana.png} & Rhode Island & \includegraphics{pics/StateFlags/RhodeIsland.png} \\\hline 
Iowa & \includegraphics{pics/StateFlags/Iowa.png} & South Carolina & \includegraphics{pics/StateFlags/SouthCarolina.png} \\\hline 
Kansas & \includegraphics{pics/StateFlags/Kansas.png} & South Dakota & \includegraphics{pics/StateFlags/SouthDakota.png} \\\hline 
Kentucky & \includegraphics{pics/StateFlags/Kentucky.png} & Tennessee & \includegraphics{pics/StateFlags/Tennessee.png} \\\hline 
Louisiana & \includegraphics{pics/StateFlags/Louisiana.png} & Texas & \includegraphics{pics/StateFlags/Texas.png} \\\hline 
Maine & \includegraphics{pics/StateFlags/Maine.png} & Utah & \includegraphics{pics/StateFlags/Utah.png} \\\hline 
Maryland & \includegraphics{pics/StateFlags/Maryland.png} & Vermont & \includegraphics{pics/StateFlags/Vermont.png} \\\hline 
Massachusetts & \includegraphics{pics/StateFlags/Massachusetts.png} & Virginia & \includegraphics{pics/StateFlags/Virginia.png} \\\hline 
Michigan & \includegraphics{pics/StateFlags/Michigan.png} & Washington & \includegraphics{pics/StateFlags/Washington.png} \\\hline 
Minnesota & \includegraphics{pics/StateFlags/Minnesota.png} & West Virginia & \includegraphics{pics/StateFlags/WestVirginia.png} \\\hline 
Mississippi & \includegraphics{pics/StateFlags/Mississippi.png} & Wisconsin & \includegraphics{pics/StateFlags/Wisconsin.png} \\\hline 
Missouri & \includegraphics{pics/StateFlags/Missouri.png} & Wyoming & \includegraphics{pics/StateFlags/Wyoming.png} \\\hline 
\end{tabular}



\end{center}

\quad \\



\begin{example}

$\Large{(}$Ohio, \includegraphics{pics/StateFlags/Ohio.png}$\Large{)}$ is a pair in the StateFlags function.

StateFlags(Ohio) = \includegraphics{pics/StateFlags/Ohio.png}.

Ohio is the only solution to StateFlags(state) = \includegraphics{pics/StateFlags/Ohio.png}.

\end{example}










\end{document}

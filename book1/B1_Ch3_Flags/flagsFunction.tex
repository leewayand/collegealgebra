\documentclass{ximera}

\input{../../preamble.tex}

\title{Flags Function}

\begin{document}

\begin{abstract}
%Stuff can go here later if we want!
\end{abstract}

\maketitle




\textbf{U.S. State Flags Function} \\

There is a connection between states and flags.  Each state can be paired with its state flag.  This connect is the basis for a relation.
In this way, each state (domain item) is paired with exactly one flag (codomain item), a function - the StateFlags function.


\begin{definition} 
\begin{itemize}
\item The domain of the StateFlags function is the set of all fifty states.
\item The codomain of the StateFlags function is the set of all fifty flags.
\item Each pair in the StateFlags function pairs a state with its state flag.
\end{itemize}
\end{definition}
\quad \\

So, far we have a perfectly good relation. But, just because someone says we have a function, doesn?t necessarily mean we do.  A little paranoia is needed here. We need to verify that it is a function.

\begin{itemize}
\item Does every state have a state flag?  Yes.
\item Does any state have more than one state flag?  No.
\item Is every state in exactly one pair?  Yes.
\end{itemize}


We have a function! 

\textbf{Note:}  The range = codomain for this function.
\quad \\




\begin{center}
\textbf{StateFlags Function} \\
Pairs represented via a table. \\
\begin{image}
\includegraphics{pics/allStateFlags_med.png}
\end{image}
\end{center}



\quad \\



\begin{example}

$\Large{(}$Ohio, \includegraphics{pics/StateFlags/Ohio.png}$\Large{)}$ is a pair in the StateFlags function.

StateFlags(Ohio) = \includegraphics{pics/StateFlags/Ohio.png}.

Ohio is the only solution to StateFlags(state) = \includegraphics{pics/StateFlags/Ohio.png}.

\end{example}

\quad \\






\begin{dialogue}
\item [\textbf{QUESTION}] If  $\Large{(}$ state, \includegraphics{pics/StateFlags/NewMexico.png} $\Large{)}$  is in the StateFlags function, then which state is the state coordinate?
\item [\textbf{ANSWER}]  The state coordinate would be New Mexico.
\end{dialogue}
\quad \\

\begin{dialogue}
\item [\textbf{QUESTION}] What is the range partner of California?
\item [\textbf{ANSWER}] \includegraphics{pics/StateFlags/NewMexico.png}
\end{dialogue}
\quad \\

\begin{dialogue}
\item [\textbf{QUESTION}] How do you pronounce StateFlags(Maine)?
\item [\textbf{ANSWER}]   StateFlags \textbf{OF} Maine.
\end{dialogue}
\quad \\

\begin{dialogue}
\item [\textbf{QUESTION}] Is Iowa a State?
\item [\textbf{ANSWER}]    Yes.
\end{dialogue}
\quad \\

\begin{dialogue}
\item [\textbf{QUESTION}] Is StateFlags(Iowa) a state?
\item [\textbf{ANSWER}]    No.
\end{dialogue}
\quad \\

\begin{dialogue}
\item [\textbf{QUESTION}] What is the value of StateFlags at Montana?
\item [\textbf{ANSWER}] \includegraphics{pics/StateFlags/Montana.png}
\end{dialogue}
\quad \\

\begin{dialogue}
\item [\textbf{QUESTION}] Are any of the states partnered with the same flag?
\item [\textbf{ANSWER}]    No.
\end{dialogue}



\quad \\



There isn't much going on here. We have two sets of things. We call them the domain and codomain. We are pairing up items from these two sets. This structure is called a relation. Relations is a pretty big topic - two sets and some connections. How much less could you have? Just about any two randomly selected sets of anything could have connections.

Sometimes this pairing has additional structure. It might happen that each domain item appears in exactly one pair. We call these relations functions.  Even this little extra structure makes a significant difference. Functions are much easier to investigate than general relations. So, we are focusing on functions.

This course is the study of functions. 

We will begin our study with functions connecting all kinds of sets, like quarterbacks or flags.  This will give us an opportunity to get familiar the types of questions involved in function analysis. Then we will narrow our attention to numeric functions. In particular, we will develop the library of elementary functions. Our goals include discovering properties and characteristics of the elementary functions and using our representational tools to investigate these characteristics.

\quad \\







\begin{question}
Evaluate $StateFlags(Georgia)$.
\begin{multipleChoice}
\choice{\includegraphics{pics/StateFlags/NewMexico.png}}
\choice[correct]{\includegraphics{pics/StateFlags/Georgia.png}}
\choice{\includegraphics{pics/StateFlags/Washington.png}}
\choice{\includegraphics{pics/StateFlags/Alabama.png}}
\end{multipleChoice}
\begin{feedback}
This is the state flag of Georgia.
\end{feedback}
\end{question}



\begin{question}
Evaluate $StateFlags(Washington)$.
\begin{multipleChoice}
\choice{\includegraphics{pics/StateFlags/RhodeIsland.png}}
\choice{\includegraphics{pics/StateFlags/NorthCaroliina.png}}
\choice{\includegraphics{pics/StateFlags/Mississippi.png}}
\choice[correct]{\includegraphics{pics/StateFlags/Washington.png}}
\end{multipleChoice}
\begin{feedback}
This is the state flag of Washington.
\end{feedback}
\end{question}



\begin{question}
\includegraphics{pics/StateFlags/RhodeIsland.png}
\begin{multipleChoice}
\choice{$StateFlags(Ohio)$}
\choice{$StateFlags(Kentucky)$}
\choice{$StateFlags(Illinois)$}
\choice[correct]{$StateFlags(Maryland)$}
\end{multipleChoice}
\begin{feedback}
This is the state flag of Maryland.
\end{feedback}
\end{question}

















\end{document}

\documentclass{ximera}

\input{../../../preamble.tex}



\outcome{Identify pairs in a relation.}


\begin{document}

\begin{definition}
  The function \dfn{Reverse} pairs a string of letters with the same string, but the letters listed in reverse order.
 
    \begin{itemize}
    \item The domain of Consonants is all strings of letters in the English alphabet.
    \item The codomain of Consonants is all strings of letters in the English alphabet.
    \item The pair (string1, string2) is a member of the Reverse function if string2 is string1 backwards.
    \end{itemize}

  
\end{definition}



\begin{exercise}


\begin{exercise}
Reverse("bike") = $\answer{"ekib"}$
\end{exercise}


\begin{exercise}
Reverse("mrots") = $\answer{"storm"}$
\end{exercise}


\begin{exercise}
Reverse("week") = $\answer{"keew"}$
\end{exercise}








\begin{exercise}

Does the range of Reverse equal the codomain.

  \begin{multipleChoice}
    \choice[correct]{Yes}
    \choice{No}
  \end{multipleChoice}
  \begin{feedback}
Any string can be reversed.
  \end{feedback}
\end{exercise}




\begin{exercise}

If string1 and string2 are two different strings of letters, then is it possible for Reverse(string1) = Reverse(string2)?

  \begin{multipleChoice}
    \choice{Yes}
    \choice[correct]{No}
  \end{multipleChoice}
  \begin{feedback}
If the reversal of two strings are equal, then the original strings have to be equal.
  \end{feedback}
\end{exercise}








\begin{exercise}

How many solutions does Reverse(string) = "school"? have? $\answer{1}$


  \begin{feedback}
"loohcs" is the only solution.
  \end{feedback}
\end{exercise}








\begin{exercise}

Is it possible for Reverse(string) = string?

  \begin{multipleChoice}
    \choice[correct]{Yes}
    \choice{No}
  \end{multipleChoice}
  \begin{feedback}
Example: Reverse("dad") = "dad"
  \end{feedback}
\end{exercise}


\begin{exercise}

Does the equation Reverse(Reverse(string)) = string have any solutions?

  \begin{multipleChoice}
    \choice[correct]{Yes}
    \choice{No}
  \end{multipleChoice}
  \begin{feedback}
The solution set is the whole domain.  The reversal of the reversal of any string results in the original string.
  \end{feedback}
\end{exercise}




\end{document}

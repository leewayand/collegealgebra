\documentclass{ximera}

\input{../../../preamble.tex}



\outcome{Identify pairs in a relation.}


\begin{document}

\begin{definition}
  The map below defines the \textbf{Flags} relation.
  
  

    \includegraphics[width=536px,height=338px]{pics/flags/flag_letter.png}

  
 

  The \dfn{Flags} relation includes three sets:
    \begin{itemize}
    \item The domain consists of the 26 flags in the table.
    \item The codomain consists of the 26 letters in the English alphabet.
    \item A set of ordered pairs. The pair (flag, letter) is a member of the Flags relation if the table pairs the flag and letter.
    \end{itemize}

  
\end{definition}



\begin{exercise}

 $\bigg($L, {\includegraphics[width=42px,height=36px]{pics/flags/L.png}}$\bigg)$ $\in$ Flags 

  \begin{multipleChoice}
    \choice{True}
    \choice[correct]{False}
  \end{multipleChoice}
  \begin{feedback}
 (flag, letter)
  \end{feedback}
\end{exercise}


\begin{exercise}
 $\bigg(${\includegraphics[width=42px,height=36px]{pics/flags/U.png}}, U$\bigg)$ $\in$ Flags 

  \begin{multipleChoice}
    \choice[correct]{True}
    \choice{False}
  \end{multipleChoice}
  \begin{feedback}
(flag, letter)
  \end{feedback}
\end{exercise}





\begin{definition}
The value of a function at a particular domain element is also called the \dfn{image} of the domain element.
  
\end{definition}

\begin{example}

The image of {\includegraphics[width=40px,height=34px]{pics/flags/T.png}} under the Flags function is T.

Flags$\bigg($ {\includegraphics[width=40px,height=34px]{pics/flags/T.png}} $\bigg)$ = T

\end{example}




\begin{definition}
  The domain of the Flags function includes 26 flags.  The domain is a \dfn{set}.  A separate collection of some of these flags would form a \dfn{subset} of the domain.
  
  Suppose D is a subset of the domain.
  
  The \dfn{Image} of D would be the subset of the range consisting of the images of all of the elements of D. The image of D under the Flags function is written as Flags(D).
 
  
\end{definition}






\begin{exercise}
Let  SomeFlags = $\Bigg\{$  {\includegraphics[width=40px,height=34px]{pics/flags/T.png}}, {\includegraphics[width=39px,height=33px]{pics/flags/W.png}}, {\includegraphics[width=40px,height=34px]{pics/flags/G.png}}  $\Bigg\}$

Determine Flags(SomeFlags)

  \begin{selectAll}
  \choice{X}
  \choice[correct]{T}
  \choice{B}
  \choice[correct]{G}
  \choice[correct]{W}
  \end{selectAll}
  \begin{feedback}
The image is the subset \{ T, W, G  \}
  \end{feedback}
\end{exercise}













\begin{definition}
  The range of the Flags function includes 26 letters.  
    
  Suppose R is a subset of the domain.
  
  The \dfn{Preimage} of R would be the subset of the domain consisting of the flags whose image is an element of R. The preimage of R under the Flags function is written as $Flags^{-1}(R)$.
 
 The -1 exponent generally invokes the ideas of inverse, reverse, backwards, upside-down, or generally opposite.
  
\end{definition}




\begin{exercise}
Let  SomeLetters = \{  letter | letter is a vowel  \}

Determine $Flags^{-1}(SomeLetters)$

  \begin{selectAll}
  \choice[correct]{{\includegraphics[width=42px,height=36px]{pics/flags/A.png}}}
  \choice[correct]{{\includegraphics[width=40px,height=34px]{pics/flags/E.png}}}
  \choice[correct]{{\includegraphics[width=39px,height=32px]{pics/flags/I.png}}}
  \choice{{\includegraphics[width=39px,height=34px]{pics/flags/K.png}}}
  \choice[correct]{{\includegraphics[width=39px,height=33px]{pics/flags/O.png}}}
  \choice[correct]{{\includegraphics[width=39px,height=34px]{pics/flags/U.png}}}
  \end{selectAll}
  \begin{feedback}
The preimage is the set of flags corresponding to \{ A, E, I, O , U \}.
  \end{feedback}
\end{exercise}













\end{document}

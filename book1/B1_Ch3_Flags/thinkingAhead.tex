\documentclass{ximera}

\input{../../preamble.tex}

\title{Thinking Ahead}

\begin{document}

\begin{abstract}
%Stuff can go here later if we want!
\end{abstract}

\maketitle




\textbf{Counterexamples} \\

Throughout this course you will encounter many functions and many statements about these functions.  Some of these statements will be true and some of them will be false.

It is not always so easy to convince someone that a statement is true.  But, it is relatively straightforward to show a statement is false - simply provide a domain-range pair for the function that doesn't satisfy the statement. Such domain-range pairs are called counterexamples.  They are examples that run counter to the statement.
\quad \\

\begin{example}

Statement : For the StateFlags function, StateFlags(state) includes the color green for every state.

A counter example to this statement is the pair $\Large{(}$Texas, ? \includegraphics{pics/StateFlags/Texas.png}$\Large{)}$.

$StateFlags(Texas)$ = \includegraphics{pics/StateFlags/Texas.png}, which does not include green.  
?
\end{example}
\quad \\


\textbf{Thinking Ahead}\ \

We began investigating functions because they were clean relations.  The domain side of the function was clean, because each domain item was only in a single pair.  The StateFlags function is also clean on the codomain side.  The State Flags function was especially clean. We might want to consider these types of functions later.  For now, we'll allow the codomain side to be cluttered.
\quad \\


\textbf{Thinking Ahead}\ \
Questions of TYPE II are often phrased in terms of solving equations. The function notation is set equal to the known codomain item and a variable represents the unknown domain item. There can be many domain items paired with a given codomain item, so the solutions to the equation are gathered together as a set, the solution set (possibly empty).

This is a very common way of posing these types of questions. Equations and inequalities present range requirements and solutions are found in the domain.  This can get quite complex as we introduce numeric functions and formulas.  Before that, we'll experience both types of questions via graphs.\quad \\


\begin{question}

The map below defines the Shirt function.
\begin{image}
\includegraphics{pics/shirt_func.png}
\end{image}
How many items are in the range?  $\answer{3}$
\begin{feedback}
The range is
\begin{image}
\includegraphics{pics/shirt_range.png}
\end{image}
\end{feedback}

\end{question}










\end{document}

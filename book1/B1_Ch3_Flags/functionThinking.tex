\documentclass{ximera}

\input{../../preamble.tex}

\title{Function Thinking}

\begin{document}

\begin{abstract}
%Stuff can go here later if we want!
\end{abstract}

\maketitle





\begin{center}
\textbf{StateFlags Function} \\
Pairs represented via a table. \\
\begin{image}
\includegraphics{pics/allStateFlags_med.png}
\end{image}
\end{center}






\begin{question}

Write function notation for \includegraphics{pics/shirt_func.png}

\begin{multipleChoice}
\choice{$StateFlags(Arizona)$}
\choice{$StateFlags(Oregon)$}
\choice{$StateFlags(Florida)$}
\choice[correct]{$StateFlags(Colorado)$}
\end{multipleChoice}
\begin{feedback}
That is the state flag of Colorado.
\end{feedback}
\end{question}




\begin{question}

Evaluate $StateFlags(Wyoming)$

\begin{multipleChoice}
\choice{\includegraphics{pics/StateFlags/Wisconsin.png}}
\choice{\includegraphics{pics/StateFlags/Wyoming.png}}
\choice{\includegraphics{pics/StateFlags/Virginia.png}}
\choice{\includegraphics{pics/StateFlags/SouthCarolina.png}}
\end{multipleChoice}
\begin{feedback}
$StateFlags(Wyoming)$ represents the range partner of Wyoming, which is Wyoming's flag.
\end{feedback}
\end{question}






\begin{question}

What is the value of StateFlags at Minnesota?

\begin{multipleChoice}
\choice{\includegraphics{pics/StateFlags/Mississippi.png}}
\choice[correct]{\includegraphics{pics/StateFlags/Missouri.png}}
\choice{\includegraphics{pics/StateFlags/Minnesota.png}}
\choice{\includegraphics{pics/StateFlags/Michigan.png}}
\end{multipleChoice}
\begin{feedback}
$StateFlags(Minnesota)$ 
\end{feedback}
\end{question}






\begin{question}

What does StateFlags of Hawaii equal

\begin{multipleChoice}
\choice{\includegraphics{pics/StateFlags/Hawaii.png}}
\choice[correct]{\includegraphics{pics/StateFlags/Iowa.png}}
\choice{\includegraphics{pics/StateFlags/Maine.png}}
\choice{\includegraphics{pics/StateFlags/Alaska.png}}
\end{multipleChoice}
\begin{feedback}
$StateFlags(Hawaii)$ 
\end{feedback}
\end{question}





\begin{question}

Solve $StateFlags(state)$ = includegraphics{pics/StateFlags/Indiana.png}

\begin{multipleChoice}
\choice{Alaska}
\choice[correct]{Indiana}
\choice{Nevada}
\choice{Louisiana}
\end{multipleChoice}
\begin{feedback}
$StateFlags(Indiana)$ = \includegraphics{pics/StateFlags/Indiana.png}
\end{feedback}
\end{question}
\quad \\



Define the Ball function as follows: //
\begin{image}
\includegraphics{pics/func_ball.png}
\end{image}

\begin{question}
Evaluate \includegraphics{pics/Q_1.png}
\begin{multipleChoice}
\choice{\includegraphics{pics/base_ball.png}}
\choice[correct]{\includegraphics{pics/basket_ball.png}}
\choice{\includegraphics{pics/soccer_ball.png}}
\choice{\includegraphics{pics/tennis_ball.png}}
\end{multipleChoice}
\begin{feedback}
\includegraphics{pics/A_1.png}
\end{feedback}
\end{question}


\begin{question}
Solve $Ball(cone)$ = \includegraphics{pics/foot_ball.png}
\begin{selectAll}
\item[correct] {\includegraphics{pics/cone_1.png}}
\item[correct] {\includegraphics{pics/cone_2.png}}
\item {\includegraphics{pics/cone_3.png}}
\item {\includegraphics{pics/cone_4.png}}
\end{selectAll}
\begin{feedback}
\includegraphics{pics/A_2.png}
\end{feedback}
\end{question}

\quad \\



The function mapped below is called a \textbf{constant function}, since all of the function values remain constant throughout the domain.
\begin{image}
\includegraphics{pics/const_func.png}
\end{image}

Let's call this function, \textbf{C}.


\begin{question}
True of False: $C(shirt_1) = C(shirt_2)$ for every $shirt_1$ and $shirt_2$ in the domain.
\begin{multipleChoice}
\item[correct] {True}
\item {False}
\end{multipleChoice}
\begin{feedback}
True.  The function has the same value for every shirt in the domain.
\end{feedback}
\end{question}










\end{document}

\documentclass{ximera}

\input{../../preamble.tex}

\title{Function Notation}

\begin{document}

\begin{abstract}
%Stuff can go here later if we want!
\end{abstract}

\maketitle




\begin{remark} \textbf{Function Notation}
The nice thing about function structure is that once you pick an item from the domain, you have automatically picked an item from the codomain, exactly one  Since each domain item is connected to exactly one codomain item, this connection automatically transfers your domain selection to a single range selection.
\end{remark}
\quad \\


If we pick Kyra Hogan in the MartinIceCream domain, then we have also automatically selected an ice cream flavor.  We may not immediately know exactly what this flavor is. We?ll have to go through the pairings to find Kyra Hogan's codomain partner.  But, we know that there is one. And, we know it is exactly one.

This function structure allows us to talk ABOUT the codomain partners without having to sift through the pairs to locate the exact value.  We can talk ABOUT Kyra Hogan's favorite ice cream flavor without knowing the exact flavor.  

Again, the phrase "Kyra Hogan's favorite ice cream flavor" is simply too long for mathematicians. And, the phrase "Kyra Hogan's codomain partner in the MartinIceCream function" is just out of the question. We need some more shorthand notation.

First, our shorthand notation needs to supply the name of the function, so that people know which package we are talking about.  Then it needs to identify the domain item, which is Kyra Hogan, here.  The traditional way of arranging these is to wrap the domain item in parentheses next to the function name.

\begin{center}
\textbf{MartinIceCream(Kyra Hogan)}
\end{center}

This shorthand notation represents an item in the codomain of the MartinIceCream function, the one paired with Kyra Hogan.  Of course, we could determine that this is Strawberry.  Then we could tell people this by writing

\begin{center}
\textbf{MartinIceCream(Kyra Hogan) = Strawberry}
\end{center}

"In the MartinIceCream function, Kyra Hogan's codomain partner is  Strawberry". If you know the pairing is for favorite flavor, then you would know that Kyra's favorite ice cream flavor is strawberry.


\begin{example}
\textbf{MartinIceCream(Mark Tortan)} represents an ice cream flavor. To determine this flavor we need to locate the pair in the MartinIceCream function that has Mark Tortan in the domain/first/left/name coordinate.

\begin{center}
(Mark Tortan, Chocolate)
\end{center}

MartinIceCream(Mark Tortan) represents the codomain partner, which sits in the codomain/second/right/value coordinate.

\begin{center}
MartinIceCream(Mark Tortan) = Chocolate    
\end{center}
\end{example}











\end{document}

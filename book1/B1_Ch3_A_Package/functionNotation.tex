\documentclass{ximera}

%\usepackage{todonotes}

\newcommand{\todo}{}

\usepackage{esint} % for \oiint
\graphicspath{
  {./}
  {ximeraTutorial/}
}

\newcommand{\mooculus}{\textsf{\textbf{MOOC}\textnormal{\textsf{ULUS}}}}

\usepackage{tkz-euclide}
\tikzset{>=stealth} %% cool arrow head
\tikzset{shorten <>/.style={ shorten >=#1, shorten <=#1 } } %% allows shorter vectors

\usetikzlibrary{backgrounds} %% for boxes around graphs
\usetikzlibrary{shapes,positioning}  %% Clouds and stars
\usetikzlibrary{matrix} %% for matrix
\usepgfplotslibrary{polar} %% for polar plots
\usetkzobj{all}
\usepackage[makeroom]{cancel} %% for strike outs
%\usepackage{mathtools} %% for pretty underbrace % Breaks Ximera
\usepackage{multicol}
\usepackage{pgffor} %% required for integral for loops


%% http://tex.stackexchange.com/questions/66490/drawing-a-tikz-arc-specifying-the-center
%% Draws beach ball 
\tikzset{pics/carc/.style args={#1:#2:#3}{code={\draw[pic actions] (#1:#3) arc(#1:#2:#3);}}}



\usepackage{array}
\setlength{\extrarowheight}{+.1cm}   
\newdimen\digitwidth
\settowidth\digitwidth{9}
\def\divrule#1#2{
\noalign{\moveright#1\digitwidth
\vbox{\hrule width#2\digitwidth}}}





\newcommand{\RR}{\mathbb R}
\newcommand{\R}{\mathbb R}
\newcommand{\N}{\mathbb N}
\newcommand{\Z}{\mathbb Z}

\newcommand{\sagemath}{\textsf{SageMath}}


%\renewcommand{\d}{\,d\!}
\renewcommand{\d}{\mathop{}\!d}
\newcommand{\dd}[2][]{\frac{\d #1}{\d #2}}
\newcommand{\pp}[2][]{\frac{\partial #1}{\partial #2}}
\renewcommand{\l}{\ell}
\newcommand{\ddx}{\frac{d}{\d x}}

\newcommand{\zeroOverZero}{\ensuremath{\boldsymbol{\tfrac{0}{0}}}}
\newcommand{\inftyOverInfty}{\ensuremath{\boldsymbol{\tfrac{\infty}{\infty}}}}
\newcommand{\zeroOverInfty}{\ensuremath{\boldsymbol{\tfrac{0}{\infty}}}}
\newcommand{\zeroTimesInfty}{\ensuremath{\small\boldsymbol{0\cdot \infty}}}
\newcommand{\inftyMinusInfty}{\ensuremath{\small\boldsymbol{\infty - \infty}}}
\newcommand{\oneToInfty}{\ensuremath{\boldsymbol{1^\infty}}}
\newcommand{\zeroToZero}{\ensuremath{\boldsymbol{0^0}}}
\newcommand{\inftyToZero}{\ensuremath{\boldsymbol{\infty^0}}}



\newcommand{\numOverZero}{\ensuremath{\boldsymbol{\tfrac{\#}{0}}}}
\newcommand{\dfn}{\textbf}
%\newcommand{\unit}{\,\mathrm}
\newcommand{\unit}{\mathop{}\!\mathrm}
\newcommand{\eval}[1]{\bigg[ #1 \bigg]}
\newcommand{\seq}[1]{\left( #1 \right)}
\renewcommand{\epsilon}{\varepsilon}
\renewcommand{\phi}{\varphi}


\renewcommand{\iff}{\Leftrightarrow}

\DeclareMathOperator{\arccot}{arccot}
\DeclareMathOperator{\arcsec}{arcsec}
\DeclareMathOperator{\arccsc}{arccsc}
\DeclareMathOperator{\si}{Si}
\DeclareMathOperator{\proj}{\vec{proj}}
\DeclareMathOperator{\scal}{scal}
\DeclareMathOperator{\sign}{sign}


%% \newcommand{\tightoverset}[2]{% for arrow vec
%%   \mathop{#2}\limits^{\vbox to -.5ex{\kern-0.75ex\hbox{$#1$}\vss}}}
\newcommand{\arrowvec}{\overrightarrow}
%\renewcommand{\vec}[1]{\arrowvec{\mathbf{#1}}}
\renewcommand{\vec}{\mathbf}
\newcommand{\veci}{{\boldsymbol{\hat{\imath}}}}
\newcommand{\vecj}{{\boldsymbol{\hat{\jmath}}}}
\newcommand{\veck}{{\boldsymbol{\hat{k}}}}
\newcommand{\vecl}{\boldsymbol{\l}}
\newcommand{\uvec}[1]{\mathbf{\hat{#1}}}
\newcommand{\utan}{\mathbf{\hat{t}}}
\newcommand{\unormal}{\mathbf{\hat{n}}}
\newcommand{\ubinormal}{\mathbf{\hat{b}}}

\newcommand{\dotp}{\bullet}
\newcommand{\cross}{\boldsymbol\times}
\newcommand{\grad}{\boldsymbol\nabla}
\newcommand{\divergence}{\grad\dotp}
\newcommand{\curl}{\grad\cross}
%\DeclareMathOperator{\divergence}{divergence}
%\DeclareMathOperator{\curl}[1]{\grad\cross #1}
\newcommand{\lto}{\mathop{\longrightarrow\,}\limits}

\renewcommand{\bar}{\overline}

\colorlet{textColor}{black} 
\colorlet{background}{white}
\colorlet{penColor}{blue!50!black} % Color of a curve in a plot
\colorlet{penColor2}{red!50!black}% Color of a curve in a plot
\colorlet{penColor3}{red!50!blue} % Color of a curve in a plot
\colorlet{penColor4}{green!50!black} % Color of a curve in a plot
\colorlet{penColor5}{orange!80!black} % Color of a curve in a plot
\colorlet{penColor6}{yellow!70!black} % Color of a curve in a plot
\colorlet{fill1}{penColor!20} % Color of fill in a plot
\colorlet{fill2}{penColor2!20} % Color of fill in a plot
\colorlet{fillp}{fill1} % Color of positive area
\colorlet{filln}{penColor2!20} % Color of negative area
\colorlet{fill3}{penColor3!20} % Fill
\colorlet{fill4}{penColor4!20} % Fill
\colorlet{fill5}{penColor5!20} % Fill
\colorlet{gridColor}{gray!50} % Color of grid in a plot

\newcommand{\surfaceColor}{violet}
\newcommand{\surfaceColorTwo}{redyellow}
\newcommand{\sliceColor}{greenyellow}




\pgfmathdeclarefunction{gauss}{2}{% gives gaussian
  \pgfmathparse{1/(#2*sqrt(2*pi))*exp(-((x-#1)^2)/(2*#2^2))}%
}


%%%%%%%%%%%%%
%% Vectors
%%%%%%%%%%%%%

%% Simple horiz vectors
\renewcommand{\vector}[1]{\left\langle #1\right\rangle}


%% %% Complex Horiz Vectors with angle brackets
%% \makeatletter
%% \renewcommand{\vector}[2][ , ]{\left\langle%
%%   \def\nextitem{\def\nextitem{#1}}%
%%   \@for \el:=#2\do{\nextitem\el}\right\rangle%
%% }
%% \makeatother

%% %% Vertical Vectors
%% \def\vector#1{\begin{bmatrix}\vecListA#1,,\end{bmatrix}}
%% \def\vecListA#1,{\if,#1,\else #1\cr \expandafter \vecListA \fi}

%%%%%%%%%%%%%
%% End of vectors
%%%%%%%%%%%%%

%\newcommand{\fullwidth}{}
%\newcommand{\normalwidth}{}



%% makes a snazzy t-chart for evaluating functions
%\newenvironment{tchart}{\rowcolors{2}{}{background!90!textColor}\array}{\endarray}

%%This is to help with formatting on future title pages.
\newenvironment{sectionOutcomes}{}{} 



%% Flowchart stuff
%\tikzstyle{startstop} = [rectangle, rounded corners, minimum width=3cm, minimum height=1cm,text centered, draw=black]
%\tikzstyle{question} = [rectangle, minimum width=3cm, minimum height=1cm, text centered, draw=black]
%\tikzstyle{decision} = [trapezium, trapezium left angle=70, trapezium right angle=110, minimum width=3cm, minimum height=1cm, text centered, draw=black]
%\tikzstyle{question} = [rectangle, rounded corners, minimum width=3cm, minimum height=1cm,text centered, draw=black]
%\tikzstyle{process} = [rectangle, minimum width=3cm, minimum height=1cm, text centered, draw=black]
%\tikzstyle{decision} = [trapezium, trapezium left angle=70, trapezium right angle=110, minimum width=3cm, minimum height=1cm, text centered, draw=black]


\title{Function Notation}

\begin{document}

\begin{abstract}
%Stuff can go here later if we want!
\end{abstract}

\maketitle




\begin{remark} \textbf{Function Notation} \\
The nice thing about function structure is that once you pick an item from the domain, you have automatically picked an item from the codomain, exactly one  Since each domain item is connected to exactly one codomain item, this connection automatically transfers your domain selection to a single range selection.
\end{remark}
\quad \\


If we pick Kyra Hogan in the MartinIceCream domain, then we have also automatically selected an ice cream flavor.  We may not immediately know exactly what this flavor is. We'll have to go through the pairings to find Kyra Hogan's codomain partner.  But, we know that there is one. And, we know it is exactly one.

This function structure allows us to talk ABOUT the codomain partners without having to sift through the pairs to locate the exact value.  We can talk ABOUT Kyra Hogan's favorite ice cream flavor without knowing the exact flavor.  

Again, the phrase "Kyra Hogan's favorite ice cream flavor" is simply too long for mathematicians. And, the phrase "Kyra Hogan's codomain partner in the MartinIceCream function" is just out of the question. We need some more shorthand notation.

First, our shorthand notation needs to supply the name of the function, so that people know which package we are talking about.  Then it needs to identify the domain item, which is Kyra Hogan, here.  The traditional way of arranging these is to wrap the domain item in parentheses next to the function name.

\begin{center}
\textbf{MartinIceCream(Kyra Hogan)}
\end{center}

This shorthand notation represents an item in the codomain of the MartinIceCream function, the one paired with Kyra Hogan.  Of course, we could determine that this is Strawberry.  Then we could tell people this by writing

\begin{center}
\textbf{MartinIceCream(Kyra Hogan) = Strawberry}
\end{center}

"In the MartinIceCream function, Kyra Hogan's codomain partner is  Strawberry". If you know the pairing is for favorite flavor, then you would know that Kyra's favorite ice cream flavor is strawberry.


\begin{example} \textbf{MartinIceCream(Mark Tortan)} \\
\textbf{MartinIceCream(Mark Tortan)} represents an ice cream flavor. To determine this flavor we need to locate the pair in the MartinIceCream function that has Mark Tortan in the domain/first/left/name coordinate.

\begin{center}
(Mark Tortan, Chocolate)
\end{center}

MartinIceCream(Mark Tortan) represents the codomain partner, which sits in the codomain/second/right/value coordinate.

\begin{center}
MartinIceCream(Mark Tortan) = Chocolate    
\end{center}
\end{example}



\textbf{Martin Ice Cream function} \\
\begin{center}

\begin{tabular}{|l|l|l|l|}
\hline
Kindergartner & Ice Cream & Kindergartner & Ice Cream \\\hline 
Johnny Warner & Chocolate & Bridget London & Mint Chocolate \\\hline 
Sara Turns & Cookie Dough & Shawn Trails & Chocolate \\\hline 
Lisa O'Reilly & Vanilla & Quinn Wallace & Chocolate \\\hline 
Martha Reater & Vanilla & Brady Delter & Strawberry \\\hline 
Ashley Teasley & Cookie Dough & Amy Ward & Cookie Dough \\\hline 
Mark Tortan & Chocolate & Micah McCain & Vanilla \\\hline 
Silas Vane & Vanilla & Bobby Post & Chocolate Chip \\\hline 
Kyra Hogan & Strawberry & Leah Korch & Peach \\\hline 
Steven Stokes & Chocolate & Samantha Devon & Strawberry \\\hline 
Emeka Simpson & Vanilla & Charlie Allston & Chocolate \\\hline 
\end{tabular}

\end{center}

\quad \\




\begin{question}
Evaluate MartinIceCream(Shawn Trails).
\begin{multipleChoice}
\choice{Vanilla}
\choice[correct]{Chocolate}
\choice{Peach}
\choice{No Value}
\end{multipleChoice}
\end{question}



\begin{question}
What is the value of MartinIceCream at Bobby Post?
\begin{multipleChoice}
\choice{Vanilla}
\choice{Strawberry}
\choice[correct]{Chocolate Chip}
\choice{No Value}
\end{multipleChoice}
\end{question}




\begin{question}
What is the value of MartinIceCream(Billy Packers)?
\begin{multipleChoice}
\choice{Vanilla}
\choice{Peach}
\choice{Mint Chocolate}
\choice[correct]{No Value}
\end{multipleChoice}
\end{question}





\begin{remark} \textbf{COMMUNICATION} \\
\quad \\
When reading our shorthand function notation people say "OF".

MartinIceCream(Mark Tortan) represents a codomain item and is read as ?Martin Ice Cream \textbf{OF} Mark Tortan.? When written this way, people refer to this codomain partner,
?MartinIceCream(Mark Tortan), as the "\textbf{VALUE}" of the MartinIceCream function \textbf{AT} Mark Tortan.

Sometimes people will say "the value of MartinIceCream at Mark Tortan is Chocolate."
MartinIceCream(Mark Tortan) is also referred to as the \textbf{IMAGE} of Mark Tortan under MartinIceCream.

When people hunt down the exact codomain partner for a given domain item, they will also use the word "EVALUATE".
"Evaluate MartinIceCream at Mark Tortan."

"Chocolate is the value of Mark Tortan under the MartinIceCream function."

We have this collection of terms, because many people in many different countries over many centuries have work on the idea of relations and functions.  So, we have many ways of talking about them. Just like everyday language.

We also extend image to a subset of the domain.  The image of a subset of the domain is the collection of all function values for all of the elements in this subset.

And, we can work in reverse.

The \textbf{PREIMAGE} of a subset of the codomain is the collection of all domain items whose function values are inside this set


\end{remark}
\quad \\





\end{document}

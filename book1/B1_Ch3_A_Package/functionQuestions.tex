\documentclass{ximera}

\input{../../preamble.tex}

\title{Function Questions}

\begin{document}

\begin{abstract}
%Stuff can go here later if we want!
\end{abstract}

\maketitle





\textbf{Pairs in a Function} \\


\begin{center}
\textbf{$(domainItem, functionName(domainItem))$}
\end{center}

The left coordinate is a domain item and the right coordinate is THE codomain partner, or THE value of the function at this domain item.
\quad \\

\begin{example}
$(Amy Ward, MartinIceCream(Amy Ward))$ is a pair in the MartinIceCream function. Amy Ward is certainly paired with $MartinIceCream(Amy Ward)$, because $MartinIceCream(Amy Ward)$ represents the codomain item paired with Amy Ward.

If we take the time to sift through the MartinIceCream table, then we might discover that Amy Ward is paired with Cookie Dough.  $(Amy Ward, Cookie Dough)$ is a pair in the MartinIceCream function. 

In fact, $(Amy Ward, MartinIceCream(Amy Ward))$ and $(Amy Ward, Cookie Dough)$ are the same pair (just written in two different ways), because $MartinIceCream(Amy Ward) = Cookie Dough$.

The equation is just our way of saying that the two expressions represent the same value.        

\end{example}


\quad \\



\textbf{Questions} \\
\quad \\

Superficially, a function is a package of pairs, which gives us just two types of questions to ask about functions.
\quad \\

\underline{TYPE I} \\
You might know the domain item and ask about its codomain partner. 

\begin{center}
(Sila Vane, flavor)
\end{center}


In this scenario, we usually just refer to the codomain partner via function notation. "flavor" is representing Sila Vane's codomain partner, so let's just write that:
\begin{center}
flavor = $MartinIceCream(Sila Vane)$
\end{center}

Here we know the domain item, Silas Vane, and we are representing the codomain partner, i.e. the value of MartinIceCream at Silas Vane.  

People throw around the word "evaluate" in this scenario when they might want someone to sift through the pairs and identify the exact codomain value for $MartinIceCream(Sila Vane)$.

\quad \\
\underline{TYPE II} \\

On the other hand, you might know the codomain item and ask about its domain partner.
\begin{center}
(name, Strawberry)
\end{center}

In this scenario we know the value of the function, which means we can use an equation to describe the value of the function.
\begin{center}
$MartinIceCream(name)$ = Strawberry 
\end{center}


This is an equation.  It has a variable, which is representing domain items. Any domain item which can replace the variable and create a true statement about the function is called a solution.  Naturally, people throw around the word ?solve? in this scenario. The solutions are the domain items that complete the pairs in question.

Here the variable name could be representing Kyra Hogan, Brady Delter, or Samantha Devon.


$name \in \{ Kyra Hogan, Brady Delter, Samantha Devon \}$

This is called the solution set of the equation.


\quad \\

\begin{itemize}
\item \textit{If you know the domain item in a pair, then there can only be one range partner.  That is how a function works.}

\item \textit{If you know the range item in the pair, then there could be multiple domain partners. The special rule of functions only only talks about the domain in pairs.}
\end{itemize}


\quad \\

\textbf{More Pieces to a Function} \\

As we saw in the MartinIceCream function each domain name had to appear in an ordered pair, but not every codomain items has to appear in an ordered pair. The rule for functions only affects the domain. Some of the codomain items are in ordered pairs and some are not. We might as well give a name to the subset of codomain items that do appear in ordered pairs.










\end{document}

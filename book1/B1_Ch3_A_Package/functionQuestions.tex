\documentclass{ximera}

\input{../../preamble.tex}

\title{Function Questions}

\begin{document}

\begin{abstract}
%Stuff can go here later if we want!
\end{abstract}

\maketitle





\textbf{Pairs in a Function} \\


\begin{center}
\textbf{$(domain_item, function_name(domain_item))$}
\end{center}

The left coordinate is a domain item and the right coordinate is THE codomain partner, or THE value of the function at this domain item.


\begin{example}
$(Amy Ward, MartinIceCream(Amy Ward))$ is a pair in the MartinIceCream function. Amy Ward is certainly paired with ?$MartinIceCream(Amy Ward)$, because ?$MartinIceCream(Amy Ward)$ represents the codomain item paired with Amy Ward.

If we take the time to sift through the MartinIceCream table, then we might discover that Amy Ward is paired with Cookie Dough.  $(Amy Ward, Cookie Dough)$ is a pair in the MartinIceCream function. 

In fact, $(Amy Ward, MartinIceCream(Amy Ward))$ and ?$(Amy Ward, Cookie Dough)$ are the same pair (just written in two different ways), because ?$MartinIceCream(Amy Ward) = Cookie Dough$.

The equation is just our way of saying that the two expressions represent the same value.        

\end{example}



bottom





\end{document}

\documentclass{ximera}

%\usepackage{todonotes}

\newcommand{\todo}{}

\usepackage{esint} % for \oiint
\graphicspath{
  {./}
  {ximeraTutorial/}
}

\newcommand{\mooculus}{\textsf{\textbf{MOOC}\textnormal{\textsf{ULUS}}}}

\usepackage{tkz-euclide}
\tikzset{>=stealth} %% cool arrow head
\tikzset{shorten <>/.style={ shorten >=#1, shorten <=#1 } } %% allows shorter vectors

\usetikzlibrary{backgrounds} %% for boxes around graphs
\usetikzlibrary{shapes,positioning}  %% Clouds and stars
\usetikzlibrary{matrix} %% for matrix
\usepgfplotslibrary{polar} %% for polar plots
\usetkzobj{all}
\usepackage[makeroom]{cancel} %% for strike outs
%\usepackage{mathtools} %% for pretty underbrace % Breaks Ximera
\usepackage{multicol}
\usepackage{pgffor} %% required for integral for loops


%% http://tex.stackexchange.com/questions/66490/drawing-a-tikz-arc-specifying-the-center
%% Draws beach ball 
\tikzset{pics/carc/.style args={#1:#2:#3}{code={\draw[pic actions] (#1:#3) arc(#1:#2:#3);}}}



\usepackage{array}
\setlength{\extrarowheight}{+.1cm}   
\newdimen\digitwidth
\settowidth\digitwidth{9}
\def\divrule#1#2{
\noalign{\moveright#1\digitwidth
\vbox{\hrule width#2\digitwidth}}}





\newcommand{\RR}{\mathbb R}
\newcommand{\R}{\mathbb R}
\newcommand{\N}{\mathbb N}
\newcommand{\Z}{\mathbb Z}

\newcommand{\sagemath}{\textsf{SageMath}}


%\renewcommand{\d}{\,d\!}
\renewcommand{\d}{\mathop{}\!d}
\newcommand{\dd}[2][]{\frac{\d #1}{\d #2}}
\newcommand{\pp}[2][]{\frac{\partial #1}{\partial #2}}
\renewcommand{\l}{\ell}
\newcommand{\ddx}{\frac{d}{\d x}}

\newcommand{\zeroOverZero}{\ensuremath{\boldsymbol{\tfrac{0}{0}}}}
\newcommand{\inftyOverInfty}{\ensuremath{\boldsymbol{\tfrac{\infty}{\infty}}}}
\newcommand{\zeroOverInfty}{\ensuremath{\boldsymbol{\tfrac{0}{\infty}}}}
\newcommand{\zeroTimesInfty}{\ensuremath{\small\boldsymbol{0\cdot \infty}}}
\newcommand{\inftyMinusInfty}{\ensuremath{\small\boldsymbol{\infty - \infty}}}
\newcommand{\oneToInfty}{\ensuremath{\boldsymbol{1^\infty}}}
\newcommand{\zeroToZero}{\ensuremath{\boldsymbol{0^0}}}
\newcommand{\inftyToZero}{\ensuremath{\boldsymbol{\infty^0}}}



\newcommand{\numOverZero}{\ensuremath{\boldsymbol{\tfrac{\#}{0}}}}
\newcommand{\dfn}{\textbf}
%\newcommand{\unit}{\,\mathrm}
\newcommand{\unit}{\mathop{}\!\mathrm}
\newcommand{\eval}[1]{\bigg[ #1 \bigg]}
\newcommand{\seq}[1]{\left( #1 \right)}
\renewcommand{\epsilon}{\varepsilon}
\renewcommand{\phi}{\varphi}


\renewcommand{\iff}{\Leftrightarrow}

\DeclareMathOperator{\arccot}{arccot}
\DeclareMathOperator{\arcsec}{arcsec}
\DeclareMathOperator{\arccsc}{arccsc}
\DeclareMathOperator{\si}{Si}
\DeclareMathOperator{\proj}{\vec{proj}}
\DeclareMathOperator{\scal}{scal}
\DeclareMathOperator{\sign}{sign}


%% \newcommand{\tightoverset}[2]{% for arrow vec
%%   \mathop{#2}\limits^{\vbox to -.5ex{\kern-0.75ex\hbox{$#1$}\vss}}}
\newcommand{\arrowvec}{\overrightarrow}
%\renewcommand{\vec}[1]{\arrowvec{\mathbf{#1}}}
\renewcommand{\vec}{\mathbf}
\newcommand{\veci}{{\boldsymbol{\hat{\imath}}}}
\newcommand{\vecj}{{\boldsymbol{\hat{\jmath}}}}
\newcommand{\veck}{{\boldsymbol{\hat{k}}}}
\newcommand{\vecl}{\boldsymbol{\l}}
\newcommand{\uvec}[1]{\mathbf{\hat{#1}}}
\newcommand{\utan}{\mathbf{\hat{t}}}
\newcommand{\unormal}{\mathbf{\hat{n}}}
\newcommand{\ubinormal}{\mathbf{\hat{b}}}

\newcommand{\dotp}{\bullet}
\newcommand{\cross}{\boldsymbol\times}
\newcommand{\grad}{\boldsymbol\nabla}
\newcommand{\divergence}{\grad\dotp}
\newcommand{\curl}{\grad\cross}
%\DeclareMathOperator{\divergence}{divergence}
%\DeclareMathOperator{\curl}[1]{\grad\cross #1}
\newcommand{\lto}{\mathop{\longrightarrow\,}\limits}

\renewcommand{\bar}{\overline}

\colorlet{textColor}{black} 
\colorlet{background}{white}
\colorlet{penColor}{blue!50!black} % Color of a curve in a plot
\colorlet{penColor2}{red!50!black}% Color of a curve in a plot
\colorlet{penColor3}{red!50!blue} % Color of a curve in a plot
\colorlet{penColor4}{green!50!black} % Color of a curve in a plot
\colorlet{penColor5}{orange!80!black} % Color of a curve in a plot
\colorlet{penColor6}{yellow!70!black} % Color of a curve in a plot
\colorlet{fill1}{penColor!20} % Color of fill in a plot
\colorlet{fill2}{penColor2!20} % Color of fill in a plot
\colorlet{fillp}{fill1} % Color of positive area
\colorlet{filln}{penColor2!20} % Color of negative area
\colorlet{fill3}{penColor3!20} % Fill
\colorlet{fill4}{penColor4!20} % Fill
\colorlet{fill5}{penColor5!20} % Fill
\colorlet{gridColor}{gray!50} % Color of grid in a plot

\newcommand{\surfaceColor}{violet}
\newcommand{\surfaceColorTwo}{redyellow}
\newcommand{\sliceColor}{greenyellow}




\pgfmathdeclarefunction{gauss}{2}{% gives gaussian
  \pgfmathparse{1/(#2*sqrt(2*pi))*exp(-((x-#1)^2)/(2*#2^2))}%
}


%%%%%%%%%%%%%
%% Vectors
%%%%%%%%%%%%%

%% Simple horiz vectors
\renewcommand{\vector}[1]{\left\langle #1\right\rangle}


%% %% Complex Horiz Vectors with angle brackets
%% \makeatletter
%% \renewcommand{\vector}[2][ , ]{\left\langle%
%%   \def\nextitem{\def\nextitem{#1}}%
%%   \@for \el:=#2\do{\nextitem\el}\right\rangle%
%% }
%% \makeatother

%% %% Vertical Vectors
%% \def\vector#1{\begin{bmatrix}\vecListA#1,,\end{bmatrix}}
%% \def\vecListA#1,{\if,#1,\else #1\cr \expandafter \vecListA \fi}

%%%%%%%%%%%%%
%% End of vectors
%%%%%%%%%%%%%

%\newcommand{\fullwidth}{}
%\newcommand{\normalwidth}{}



%% makes a snazzy t-chart for evaluating functions
%\newenvironment{tchart}{\rowcolors{2}{}{background!90!textColor}\array}{\endarray}

%%This is to help with formatting on future title pages.
\newenvironment{sectionOutcomes}{}{} 



%% Flowchart stuff
%\tikzstyle{startstop} = [rectangle, rounded corners, minimum width=3cm, minimum height=1cm,text centered, draw=black]
%\tikzstyle{question} = [rectangle, minimum width=3cm, minimum height=1cm, text centered, draw=black]
%\tikzstyle{decision} = [trapezium, trapezium left angle=70, trapezium right angle=110, minimum width=3cm, minimum height=1cm, text centered, draw=black]
%\tikzstyle{question} = [rectangle, rounded corners, minimum width=3cm, minimum height=1cm,text centered, draw=black]
%\tikzstyle{process} = [rectangle, minimum width=3cm, minimum height=1cm, text centered, draw=black]
%\tikzstyle{decision} = [trapezium, trapezium left angle=70, trapezium right angle=110, minimum width=3cm, minimum height=1cm, text centered, draw=black]


\title{A Package}

\begin{document}

\begin{abstract}
%Stuff can go here later if we want!
\end{abstract}

\maketitle





\begin{sectionOutcomes}

A function is a relation, which means it is a package.\\
It contains three sets:
\begin{itemize}
\item The DOMAIN
\item The CODOMAIN
\item The set of PAIRS
\end{itemize}
\end{sectionOutcomes}


But, a function is a special relation.  Its pairs follow a special rule: \\
Each domain item is in \underline{EXACTLY} one pair.

Let's make this official.


\begin{definition}  
A function is a package containing three sets with one rule.
\begin{itemize}
\item the Domain. 
\item the CoDomain. 
\item the set of Pairs.
\item Rule : each domain item is paired with exactly one codomain item.  i.e. is in exactly one pair.
\end{itemize}
\end{definition}



\begin{remark} COMMUNICATION
When placed inside pairs and written with parentheses, the individual items are often called coordinates. \\
Written ordered pairs look like (domain item, codomain item).
\begin{itemize}
\item All of the first (left) coordinates are items from the domain.
\item All of the second (right) coordinates are items from the codomain.
\end{itemize}
\end{remark}

\begin{observation} 
\textbf{Note:} "exactly one" means one. Not zero. Not two. Just one.
\end{observation}

\begin{observation} 
\textbf{Note:} Since the domain item is always written on the left and the codomain item is always written on the right, these pairs are often said to be "ordered pairs". This helps with communication. When reading an ordered pair for a function you can always be sure that the left item is from the domain and the right item is from the codomain.
\end{observation}


\begin{explanation} \textbf{Video}
\begin{center}
  \youtube{_ZPK9TeG-SI}
\end{center}
\end{explanation}

\quad \\


\textbf{Favorite Ice Cream Flavor Function} \\

Mrs. Sadler's kindergarten class is going for ice cream today.  In preparation, each kindergartener has picked their favorite ice cream flavor. It looks like we have a connection between kindergarteners and ice cream flavors.  We have a relation.  The domain consists of the kindergarteners in Mrs. Sadler's class.  The codomain is a set of ice cream flavors. Is it a function?

It turns out this is not a function.  It turns out Bill Parker doesn't like ice cream at all. He has no favorite flavor, which means he is not paired with any flavor.  We have a domain item that is not in exactly one pair.

That seems like a technical reason for not having a function.  But, rules are rules.  

Let's move onto Mr. Martin's kindergarten class.

\begin{center}

\begin{tabular}{|l|l|l|l|}
\hline
Kindergartner & Ice Cream & Kindergartner & Ice Cream \\\hline 
Johnny Warner & Chocolate & Bridget London & Mint Chocolate \\\hline 
Sara Turns & Cookie Dough & Shawn Trails & Chocolate \\\hline 
Lisa O'Reilly & Vanilla & Quinn Wallace & Chocolate \\\hline 
Martha Reater & Vanilla & Brady Delter & Strawberry \\\hline 
Ashley Teasley & Cookie Dough & Amy Ward & Cookie Dough \\\hline 
Mark Tortan & Chocolate & Micah McCain & Vanilla \\\hline 
Silas Vane & Vanilla & Bobby Post & Chocolate Chip \\\hline 
Kyra Hogan & Strawberry & Leah Korch & Peach \\\hline 
Steven Stokes & Chocolate & Samantha Devon & Strawberry \\\hline 
Emeka Simpson & Vanilla & Charlie Allston & Chocolate \\\hline 
\end{tabular}

\end{center}

\quad \\
\textbf{Martin Ice Cream function} \\

The domain of the MartinIceCream function is the collection of all of the kindergartener's in Mr. Martin's class: 

\textit{Johnny Warner, Sarah Turns, Lisa, O?Reilly, Martha Reater, Ashley Teasler, Mark Tortan, Silas Vane, Kyra Hogan, Steven Stokes, Emeka Simpson, Bridget London, Shawn Trails, Quinn Wallace, Brady Delter, Amy Ward, Micah McCain, Bobby Post, Leah Korch, Samantha Devon, Charlie Allston}


The codomain of the MartinIceCream function consists of the ice flavors available in the school cafeteria:
\textit{Chocolate, Toffee, Cookie Dough, Vanilla, Strawberry, Mint Chocolate, Chocolate Chip, Peach, Pistachio, and Butter Pecan.}

The pairs for the MartinIceCream function are presented in the table. First, a little paranoia. Let's double check that this is a function.
Is each student in a pair? Yes. Is each student in exactly one pair? Yes. Ok, we can continue.



\quad \\



\begin{dialogue}
\item[QUESTION:] Is Bill Parker in the domain of the MartinIceCream function?
\item[ANSWER:]  No. He was in Mrs. Sadler class.
\item[QUESTION:]  The pair (Steven Stokes, Vanilla) is certainly built from a kindergartener in Mr. Martin?s class and an ice cream flavor, but is it in the MartinIceCream function?
\item[ANSWER:] No. Vanilla was not Steven?s favorite flavor.
\end{dialogue}



\begin{question}
Is (Steven Stokes, Chocolate) an ordered pair in the MartinIceCream function?
\begin{multipleChoice}
\choice{Yes}
\choice[correct]{No}
\end{multipleChoice}
\end{question}



\begin{question}
Is (Amy Ward, Cookie Dough) an ordered pair in the MartinIceCream function?
\begin{multipleChoice}
\choice[correct]{Yes}
\choice{No}
\end{multipleChoice}
\end{question}


\begin{question}
Is (Vanilla, Silas Vane) an ordered pair in the MartinIceCream function?
\begin{multipleChoice}
\choice{Yes}
\choice[correct]{No}
\end{multipleChoice}
\end{question}


\begin{question}
Is (every domain name in a pair in the MartinIceCream function?
\begin{multipleChoice}
\choice[correct]{Yes}
\choice{No}
\end{multipleChoice}
\end{question}








\end{document}

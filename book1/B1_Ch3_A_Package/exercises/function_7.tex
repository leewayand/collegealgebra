\documentclass{ximera}

\input{../../../preamble.tex}



\outcome{Identify pairs in a relation.}


\begin{document}

\begin{definition}
  The map below defines the \textbf{Ball} relation. 
  
  

    \includegraphics[width=293px,height=214px]{pics/r28.png}

  
 

  The \dfn{Ball} relation includes three sets:
    \begin{itemize}
    \item The domain consists of 5 family members identified in the map.
    \item The codomain consists of the 7 balls identified in the map.
    \item A set of ordered pairs. The pair (person, ball) is a member of the Ball relation if the person is connected to the ball with an arrow.
    \end{itemize}

  
  
\end{definition}






\begin{exercise}
How many items are in the domain of Ball? $\answer{5}$
  \begin{feedback}
5 family members.
  \end{feedback}
\end{exercise}




\begin{exercise}
How many pairs are in Ball? $\answer{5}$
  \begin{feedback}
Each domain person is in a pair.
  \end{feedback}
\end{exercise}














\begin{exercise}
Is Ball a well-defined function?
  \begin{multipleChoice}
    \choice[correct]{Yes}
    \choice{No}
  \end{multipleChoice}
  \begin{feedback}
Each domain person is associated with exactly one codomain ball.
  \end{feedback}
\end{exercise}






\begin{exercise}
There are the same number of domain items as pairs in the Ball function.
  \begin{multipleChoice}
    \choice[correct]{True}
    \choice{False}
  \end{multipleChoice}
  \begin{feedback}
Each domain person is in exactly one pair.
  \end{feedback}
\end{exercise}






\begin{definition}
  A function is said to be \textbf{onto} if the range equals the codomain. In other words, if every item in the codomain appears as a partner in some pair.   
  
\end{definition}



\begin{exercise}
 Is Ball an onto function?
  \begin{multipleChoice}
    \choice{Yes}
    \choice[correct]{No}
  \end{multipleChoice}
  \begin{feedback}
Some balls in the codomain are not included in function pairs, like the tennis ball.
  \end{feedback}
\end{exercise}




\end{document}

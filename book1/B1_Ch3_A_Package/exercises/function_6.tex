\documentclass{ximera}

\input{../../../preamble.tex}



\outcome{Identify pairs in a relation.}


\begin{document}

\begin{definition}
  The map below defines the \textbf{Cone} relation. 
  
  

    \includegraphics[width=292px,height=214px]{pics/r26.png}

  
 

  The \dfn{Cone} relation includes three sets:
    \begin{itemize}
    \item The domain consists of 1 ice cream cone identified in the map.
    \item The codomain consists of the 3 family members identified in the map.
    \item A set of ordered pairs. The pair (cone, person) is a member of the Cone relation if the cone is connected to the person with an arrow.
    \end{itemize}

  
  
\end{definition}






\begin{exercise}
How many items are in the domain of Cone? $\answer{1}$
  \begin{feedback}
Just one chocolate cone.
  \end{feedback}
\end{exercise}




\begin{exercise}
How many pairs are in Cone? $\answer{1}$
  \begin{feedback}
Just one chocolate cone.
  \end{feedback}
\end{exercise}














\begin{exercise}
Is Cone a well-defined function?
  \begin{multipleChoice}
    \choice[correct]{Yes}
    \choice{No}
  \end{multipleChoice}
  \begin{feedback}
Each domain cone is associated with exactly one codomain person.
  \end{feedback}
\end{exercise}







\begin{exercise}
How many items are in the range of Cones? $\answer{1}$
  \begin{feedback}
Only one of the three persons has an arrow pointing to her.
  \end{feedback}
\end{exercise}








\end{document}

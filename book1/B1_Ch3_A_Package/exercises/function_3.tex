\documentclass{ximera}

\input{../../../preamble.tex}



\outcome{Identify pairs in a relation.}


\begin{document}

\begin{definition}
  The map below defines the \textbf{Blue} relation. Blue is a function, since each item in the domain is partnered with exactly one item in the codomain.
  
  

    \includegraphics[width=293px,height=214px]{pics/r12.png}

  
 

  The \dfn{Blue} function is a \textbf{constant} function.  Every value of Blue is the same or its value stays constant throughout the domain.

  
\end{definition}



\begin{exercise}

 ({\includegraphics[width=28px,height=27px]{pics/elements/shirts/shirts7.png}}, {\includegraphics[width=28px,height=29px]{pics/elements/shirts/shirts3.png}}) $\in$ Blue 

  \begin{multipleChoice}
    \choice[correct]{True}
    \choice{False}
  \end{multipleChoice}
  \begin{feedback}
({\includegraphics[width=27px,height=28px]{pics/elements/shirts/shirts1.png}},  {\includegraphics[width=33px,height=49px]{pics/elements/family/family2.png}})
  \end{feedback}
\end{exercise}






\begin{exercise}
Is Blue a well-defined function?
  \begin{multipleChoice}
    \choice[correct]{Yes}
    \choice{No}
  \end{multipleChoice}
  \begin{feedback}
Each domain shirt is associated with exactly one codomain shirt.
  \end{feedback}
\end{exercise}





\begin{exercise}
How many items are in the range of Blue? $\answer{1}$
  \begin{feedback}
All arrows point to one codomain shirt.
  \end{feedback}
\end{exercise}







\begin{exercise}
 Blue({\includegraphics[width=28px,height=28px]{pics/elements/shirts/shirts2.png}}) = Blue({\includegraphics[width=28px,height=29px]{pics/elements/shirts/shirts4.png}}) 
  \begin{multipleChoice}
    \choice[correct]{True}
    \choice{False}
  \end{multipleChoice}
  \begin{feedback}
Both expressions evaluate to the same codomain value.  They are equal.
  \end{feedback}
\end{exercise}











\end{document}

\documentclass{ximera}

\input{../../../preamble.tex}



\outcome{Identify pairs in a relation.}


\begin{document}

\begin{definition}
  The map below defines the \textbf{Driver} relation. 
  
  

    \includegraphics[width=293px,height=213px]{pics/r29.png}

  
 

  The \dfn{Driver} relation includes three sets:
    \begin{itemize}
    \item The domain consists of 2 cars identified in the map.
    \item The codomain consists of the 4 drivers identified in the map.
    \item A set of ordered pairs. The pair (car, driver) is a member of the Driver relation if the car is connected to the person with an arrow.
    \end{itemize}

  
  
\end{definition}













\begin{exercise}
Is Driver a well-defined function?
  \begin{multipleChoice}
    \choice[correct]{Yes}
    \choice{No}
  \end{multipleChoice}
  \begin{feedback}
Each domain car is associated with exactly one codomain driver.
  \end{feedback}
\end{exercise}






\begin{exercise}
There are the same number of domain items as pairs in the Driver function.
  \begin{multipleChoice}
    \choice[correct]{True}
    \choice{False}
  \end{multipleChoice}
  \begin{feedback}
Each domain car is in exactly one pair.
  \end{feedback}
\end{exercise}






\begin{definition}
  A function is said to be \textbf{one-to-one} if every pair contains a different range item.   In other words, there are no repeated function values.
  
\end{definition}



\begin{exercise}
 Is Ball a one-to-one function?
  \begin{multipleChoice}
    \choice[correct]{Yes}
    \choice{No}
  \end{multipleChoice}
  \begin{feedback}
Two different drivers in the two pairs.
  \end{feedback}
\end{exercise}




\end{document}

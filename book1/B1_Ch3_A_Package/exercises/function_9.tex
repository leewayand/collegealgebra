\documentclass{ximera}

\input{../../../preamble.tex}



\outcome{Identify pairs in a relation.}


\begin{document}

\begin{definition}
  The map below defines the \textbf{Owner} relation. 
  
  

    \includegraphics[width=292px,height=213px]{pics/r30.png}

  
 

  The \dfn{Owner} relation includes three sets:
    \begin{itemize}
    \item The domain consists of 5 owners identified in the map.
    \item The codomain consists of 2 cars identified in the map.
    \item A set of ordered pairs. The pair (person, car) is a member of the Owner relation if the person is connected to the car with an arrow.
    \end{itemize}

  
  
\end{definition}













\begin{exercise}
Is Owner a well-defined function?
  \begin{multipleChoice}
    \choice[correct]{Yes}
    \choice{No}
  \end{multipleChoice}
  \begin{feedback}
Each domain owner is associated with exactly one codomain car.
  \end{feedback}
\end{exercise}






\begin{exercise}
There are the same number of domain items as pairs in the Owner function.
  \begin{multipleChoice}
    \choice[correct]{True}
    \choice{False}
  \end{multipleChoice}
  \begin{feedback}
Each domain person is in exactly one pair.
  \end{feedback}
\end{exercise}






\begin{definition}
  A function is said to be \textbf{one-to-one} if every pair contains a different range item.   In other words, there are no repeated function values.
  
\end{definition}



\begin{exercise}
 Is Owner a one-to-one function?
  \begin{multipleChoice}
    \choice{Yes}
    \choice[correct]{No}
  \end{multipleChoice}
  \begin{feedback}
\includegraphics[width=68px,height=32px]{pics/cars/cars3.png} is in three pairs.
  \end{feedback}
\end{exercise}




\end{document}

\documentclass{ximera}

\input{../../preamble.tex}

\title{NFL Quarterbacks}

\begin{document}

\begin{abstract}
%Stuff can go here later if we want!
\end{abstract}

\maketitle

\begin{sectionOutcomes}

After completing this section, students should understand relations and functions as mathematical tools. 

\begin{itemize}
\item Students .
\item Students .
\end{itemize}

\end{sectionOutcomes}





We love our sports. We follow our favorite teams and debate endlessly over comparisons, performances, and decisions. We are debating structure.  This is sports science. And, of course, there is sports mathematics describing the structure we discover.

The National Football League (NFL) dates back to the late nineteenth century.  The game has undergone many changes including rule changes, team changes, and of course, player changes. The quarterback position has seen many players throughout the years. We might say there is a relation here between quarterback players and teams.

Let?s describe a relation between NFL quarterbacks and NFL teams.  



\begin{definition}
\quad \\
A player-team pair will be a member of the \textbf{Q-Team} relation if that player filled a quarterback position on that team at any time in NFL history for any amount of time.  

\begin{itemize}
\item The first set is quarterbacks.
\item The second set is football teams.
\item The pairs look like (quarterback, team)
\end{itemize}

\end{definition}


\begin{example}
\begin{itemize}
 \item (Joe Montana, San Francisco 49ers) $\in$ Q-Team
 \item (Brett Favre, Green Bay Packers) $\in$ Q-Team
 \item (Otto Graham, Cleveland Browns) $\in$ Q-Team
\end{itemize}
\end{example}

\quad \\



\textbf{QUESTION:} Is Q-Team reflexive? \\
\textbf{ANSWER:} This doesn?t even make any sense.  You can?t have a pair like (Johnny Unitas, Johnny Unitas). You can?t have a pair like (Detroit Lions, Detroit Lions).  The pairs consist of one player and one team. So, no. Q-Team is not reflexive.


\textbf{QUESTION:}Is Q-Team symmetric?\\
If (Joe Montana, San Francisco 49ers) $\in$ Q-Team, then is it automatic that (San Francisco 49ers, Joe Montana) $\in$ Q-Team\\
\textbf{ANSWER:} Hopefully not. \\
In the Totman relation, the pairs were chosen from the same set.  So, when they were reversed, the new pair still made sense.  But, it is going to get confusing to allow pairs to be written in any order for other relations. We should add some new notation rules to help us with this.


\textbf{New Rule} \\
A relation is a bunch of pairs representing a connection between sets.  The pairs are made by taking one item from one set and a second item from another set. In some relations, both sets might be the same.  The new rule says stick with one order.  Just say it up front and then stick with it.  That will alleviate a lot of confusion.



\begin{remark} \textbf{COMMUNICATION} \\
We'll call the first set of things the \textbf{domain} of the relation. \\
We'll call the second set of things the textbf{codomain} of the relation.  \\
Pairs will always be written as textbf{(domain item, codomain item)}.
\end{remark}







\end{document}

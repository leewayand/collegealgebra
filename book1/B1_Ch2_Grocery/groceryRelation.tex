\documentclass{ximera}

\input{../../preamble.tex}

\title{The Grocery Store}

\begin{document}

\begin{abstract}
%Stuff can go here later if we want!
\end{abstract}

\maketitle

\begin{sectionOutcomes}

After completing this section, students should understand relations and functions as mathematical tools. 

\begin{itemize}
\item Students .
\item Students .
\end{itemize}

\end{sectionOutcomes}





We run our daily lives by our understanding the connections between things.  There is no better example of this than the grocery store.  While shopping, we understand that there is a connection between products and prices. We often use this product-price relation to make shopping decisions.  Let's investigate this product-price relation.

Just like with NFL quarterbacks, this is a big relation with lots of pairs, too many pairs.  We'll simplify a bit to make it more manageable. 

\begin{itemize}
\item We'll focus on a single store : Bob's Groceries
\item We'll focus on a single day : Aug 12, 2015
\end{itemize}


\begin{definition}
\textbf{The Product-Price Relation}
\begin{itemize}
\item The domain of the Product-Price Relation is the collection of all products in Bob's Groceries on Aug 12, 2015. 
\item The codomain of the Product-Price Relation is the collection of all prices. 
\item A (product, price) pair is in the Product-Price Relation if this product had this price in Bob's Groceries on Aug 12, 2015. Otherwise, it is not in the Product-Price Relation.
\end{itemize}
\end{definition}


\quad \\



\begin{example}
\begin{itemize}
\item (Freshair Wheat Bread, \$1.19) $\in$ Product-Price
\item (Greenforest Yellow Corn, \$0.89) $\in$ Product-Price
\item (Greenforest Yellow Corn, \$0.79) $\in$ Product-Price
 \end{itemize}
 \end{example}


\quad \\




\textbf{More Simplifying} 

It is possible that there is more than one size of Greenforest Yellow Corn, which could add some confusion.  Instead of writing a description of the product, let?s use the global trade identification number, i.e. its bar code number.


\begin{definition}
\textbf{The Product-Price Relation (improved)}
\begin{itemize}
\item The domain of the Product-Price Relation is the collection of all product GTIN numbers used in Bob's Groceries on Aug 12, 2015. 
\item The codomain of the Product-Price Relation is the collection of all prices. 
\item A (number, price) pair is in the Product-Price Relation if this is a GTIN number for a product in Bob's Groceries and this product had this price in Bob's Groceries on Aug 12, 2015. Otherwise, it is not in the Product-Price Relation.
\end{itemize}
\end{definition}



\begin{example}
\begin{itemize}
\item (145526374467, \$1.19) $\in$ Product-Price
\item (176228354233, \$0.89) $\in$ Product-Price
\item (176228354233, \$0.79) $\in$ Product-Price
 \end{itemize}
 \end{example}

\quad \\

\textbf{Product-Price}
To the right are all of the pairs inside the Product-Price relation.  Bob's Grocery is not a very big store. (It scrolls.)
A list is one mathematical tool for describing the connections in this relation.




Another tool is a picture map.

The map lists the domain items and the codomain items and then identifies pairs in the relation with a line segment.




\end{document}

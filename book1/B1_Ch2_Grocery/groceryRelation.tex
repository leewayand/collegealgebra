\documentclass{ximera}

\input{../../preamble.tex}

\title{The Grocery Store}

\begin{document}

\begin{abstract}
%Stuff can go here later if we want!
\end{abstract}

\maketitle

\begin{sectionOutcomes}

After completing this section, students should understand relations and functions as mathematical tools. 

\begin{itemize}
\item Students .
\item Students .
\end{itemize}

\end{sectionOutcomes}





We run our daily lives by our understanding the connections between things.  There is no better example of this than the grocery store.  While shopping, we understand that there is a connection between products and prices. We often use this product-price relation to make shopping decisions.  Let's investigate this product-price relation.

Just like with NFL quarterbacks, this is a big relation with lots of pairs, too many pairs.  We'll simplify a bit to make it more manageable. 

\begin{itemize}
\item We'll focus on a single store : Bob's Groceries
\item We'll focus on a single day : Aug 12, 2015
\end{itemize}


\begin{definition}
\textbf{The Product-Price Relation}
\begin{itemize}
\item The domain of the Product-Price Relation is the collection of all products in Bob's Groceries on Aug 12, 2015. 
\item The codomain of the Product-Price Relation is the collection of all prices. 
\item A (product, price) pair is in the Product-Price Relation if this product had this price in Bob's Groceries on Aug 12, 2015. Otherwise, it is not in the Product-Price Relation.
\end{itemize}
\end{definition}


\quad \\



\begin{example}
\begin{itemize}
\item (Freshair Wheat Bread, \$1.19) $\in$ Product-Price
\item (Greenforest Yellow Corn, \$0.89) $\in$ Product-Price
\item (Greenforest Yellow Corn, \$0.79) $\in$ Product-Price
 \end{itemize}
 \end{example}


\quad \\




\textbf{More Simplifying} 

It is possible that there is more than one size of Greenforest Yellow Corn, which could add some confusion.  Instead of writing a description of the product, let?s use the global trade identification number, i.e. its bar code number.


\begin{definition}
\textbf{The Product-Price Relation (improved)}
\begin{itemize}
\item The domain of the Product-Price Relation is the collection of all product GTIN numbers used in Bob's Groceries on Aug 12, 2015. 
\item The codomain of the Product-Price Relation is the collection of all prices. 
\item A (number, price) pair is in the Product-Price Relation if this is a GTIN number for a product in Bob's Groceries and this product had this price in Bob's Groceries on Aug 12, 2015. Otherwise, it is not in the Product-Price Relation.
\end{itemize}
\end{definition}



\begin{example}
\begin{itemize}
\item (145526374467, \$1.19) $\in$ Product-Price
\item (176228354233, \$0.89) $\in$ Product-Price
\item (176228354233, \$0.79) $\in$ Product-Price
 \end{itemize}
 \end{example}

\quad \\

\textbf{Product-Price}
To the right are all of the pairs inside the Product-Price relation.  Bob's Grocery is not a very big store. (It scrolls.)
A list is one mathematical tool for describing the connections in this relation.

\[
\begin{array}{|l|l|l|}
\hline
(102707849467, 1.24) & (151603090297, 1.28) & (147882057258, 1.33) \\\hline
(121219574901, 3.95) & (196997564303, 4.61) & (120984686834, 1.39) \\\hline
(192107524489, 3.06) & (116532522722, 5.46) & (190132246907, 1.24) \\\hline
(105853328761, 1.33) & (185066339518, 1.25) & (178355581117, 4.81) \\\hline 
(139647551818, 0.79) & (182604889507, 2.07) & (163639788196, 0.95) \\\hline 
(151593345667, 2.07) & (173630558604, 1.24) & (141821520319, 1.25) \\\hline 
(142144767669, 3.38) & (187266863018, 2.63) & (146494777598, 1.33) \\\hline 
(155152532584, 3.06) & (159664504967, 0.79) & (137707844901, 3.67) \\\hline 
(116506724164, 2.17) & (139411323037, 1.71) & (127419334209, 1.90) \\\hline 
(120957405055, 4.66) & (133860480889, 1.39) & (145242561846, 0.79) \\\hline 
(105756538007, 0.95) & (162440197320, 1.46) & (198053092474, 1.39) \\\hline 
(130874077288, 1.39) & (136348437875, 2.46) & (147817810438, 4.66) \\\hline 
(133727371104, 0.79) & (182108962832, 6.79) & (158625940456, 2.68) \\\hline 
(100883301040, 3.07) & (103811801019, 2.07) & (185611457154, 3.06) \\\hline 
(184348593986, 1.33) & (135757760868, 2.40) & (187006291505, 3.38) \\\hline 
(143404118258, 5.94) & (185248264403, 0.79) & (116047586435, 1.39) \\\hline 
(189778504216, 1.25) & (187266173278, 1.66) &    \\\hline
\end{array}
\]



Another tool is a picture map.
\begin{image}
\includegraphics{pics/Product_Price_Map.png}
\end{image}

The map lists the domain items and the codomain items and then identifies pairs in the relation with a line segment.



GTIN numbers are twelve digit numbers, but not every twelve digit number is a GTIN number. That means not every twelve digit number is in the domain the Product-Price relation.

It could also happen that the twelve digit number is a GTIN number, but Bob?s Groceries might not carry that product.  Again the number would not be in the domain.

This will be a common theme with our functions. By describing the domain as GTIN numbers we know that we are dealing with 12-digit numbers, but not all 12-digit numbers. Our description is implying which 12-digit numbers are in the domain.  We say that the domain is implied.

\begin{definition}
An implied domain is one that is not stated explicitly. It is inferred from the description of the relation.
\end{definition}













\end{document}

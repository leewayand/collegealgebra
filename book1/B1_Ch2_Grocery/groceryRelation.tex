\documentclass{ximera}

\input{../../preamble.tex}

\title{The Grocery Store}

\begin{document}

\begin{abstract}
%Stuff can go here later if we want!
\end{abstract}

\maketitle

\begin{sectionOutcomes}

After completing this section, students should understand relations and functions as mathematical tools. 

\begin{itemize}
\item Students .
\item Students .
\end{itemize}

\end{sectionOutcomes}


\Large{The Product-Price Relation}



We run our daily lives by our understanding the connections between things.  There is no better example of this than the grocery store.  While shopping, we understand that there is a connection between products and prices. We often use this product-price relation to make shopping decisions.  Let?s investigate this product-price relation.

Just like with NFL quarterbacks, this is a big relation with lots of pairs, too many pairs.  We?ll simplify a bit to make it more manageable. 

\begin{itemize}
\item We?ll focus on a single store : Bob?s Groceries
\item We?ll focus on a single day : Aug 12, 2015
\end{itemize}


\begin{definition}
\textbf{The Product-Price Relation}
\begin{itemize}
\item The domain of the Product-Price Relation is the collection of all products in Bob?s Groceries on Aug 12, 2015. 
\item The codomain of the Product-Price Relation is the collection of all prices. 
\item A (product, price) pair is in the Product-Price Relation if this product had this price in Bob?s Groceries on Aug 12, 2015. Otherwise, it is not in the Product-Price Relation.
\end{itemize}
\end{definition}





\end{document}

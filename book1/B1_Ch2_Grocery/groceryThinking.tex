\documentclass{ximera}

\input{../../preamble.tex}

\title{Grocery Thinking}

\begin{document}

\begin{abstract}
%Stuff can go here later if we want!
\end{abstract}

\maketitle


\begin{question}

Define a new function as follows: \\
Domain consists of the odd natural numbers less than 21.\\
Codomain is all natural numbers.\\
Each number in the domain is paired with its double from the codomain.

Is this a function?
\begin{multipleChoice}
\choice[correct]{Yes}
\choice{No}
\end{multipleChoice}
\begin{feedback}
Yes. The whole relation can be written as follows.
\{ (1, 2), (3, 6), (5, 10), (7, 14), (9, 18), (11, 22), (13, 26), (15, 30), (17, 34), (19, 38) \}
Each domain number is in exactly one pair.
\end{feedback}


\end{question}


\quad \\



\begin{question}

Define a new function as follows: \\
Domain consists of all real numbers. \\
Codomain is the set \{ odd, even \}. \\
Each number in the domain is paired with its parity - whether it is odd or even.

Is this a function?
\begin{multipleChoice}
\choice{Yes}
\choice[correct]{No}
\end{multipleChoice}
\begin{feedback}
No. $\sqrt{5}$ is a real number. But, it is neither odd nor even. So, it doesn?t have a partner from the codomain. So, it isn?t in a pair. But, every domain number has to be in exactly one pair.
Therefore, this is not a function. It is not well-defined.
\end{feedback}


\end{question}

\quad \\





\begin{question}

Define the State Capital relation as follows. \\?Domain is all 50 states. \\ ?Codomain is all cites in U.S. \\?A pair (state, city) is in State Capital if the city is the capital of this state.


Is (South Dakota, Pierre) in State Capital?
\begin{multipleChoice}
\choice[correct]{Yes}
\choice{No}
\end{multipleChoice}
\begin{feedback}
Pierre is the capital of South Dakota.
\end{feedback}


Is (Ohio, Cleveland) in State Capital?
\begin{multipleChoice}
\choice{Yes}
\choice[correct]{No}
\end{multipleChoice}
\begin{feedback}
Cleveland is not the capital of Ohio.
\end{feedback}


Is State Capital a function?
\begin{multipleChoice}
\choice[correct]{Yes}
\choice{No}
\end{multipleChoice}
\begin{feedback}
Yes. Each state has exactly one state capital city.
\end{feedback}

\end{question}



\quad \\


\begin{question}

Define the Rotate Left relation as follows. \\?Domain is all alphabet strings. \\?Codomain is all alphabet strings. \\?A pair (string, string) is in Rotate Left if the second string is made by rotating the letters in the first string to the right and carrying over the last letter to the first position.


Is (abcd, dabc) in Rotate Left?
\begin{multipleChoice}
\choice{Yes}
\choice[correct]{No}
\end{multipleChoice}
\begin{feedback}
No. The reverse of abcd is dcba.
\end{feedback}


Is (xyz, zyx) in Rotate Left?
\begin{multipleChoice}
\choice[correct]{Yes}
\choice{No}
\end{multipleChoice}
\begin{feedback}
These strings are the reverse of each other.
\end{feedback}


Are there any pairs in Rotate Left that hold the same string?
\begin{multipleChoice}
\choice[correct]{Yes}
\choice{No}
\end{multipleChoice}
\begin{feedback}
Yes. For example, (www, www)
\end{feedback}

\end{question}












\end{document}

\documentclass{ximera}

\input{../../preamble.tex}

\title{Mathlet: Shift Left}

\begin{document}

\begin{abstract}
%Stuff can go here later if we want!
\end{abstract}

\maketitle





Given two sets and connections between their members, we always have a relation. But, we don?t always have a function. The extra rule can be a kink at times. 


\begin{example}
Here is a diagram of two sets and line segments illustrating the connections.

\begin{center}
\begin{image}
\includegraphics{pics/func.png}
\end{image}
\end{center}



We naturally read left to right, so it is common to see \includegraphics{pics/circle.png} as the domain and \includegraphics{pics/triangle.png} as the range.  The pairs would be 
\[
\{ (A, 3), (B, 3), (C, 4) \}
\]


But, we could just as easily see \includegraphics{pics/triangle.png} as the domain and \includegraphics{pics/circle.png} as the range.  Then the pairs would be
\[
\{ (3, A), (3, B), (4, C) \}
\]


Either one is ok. Both are relations. \\
However, they are not both functions. \\
\quad \\

The pairs $\{ (A, 3), (B, 3), (C, 4) \}$ follow the rule for a function. \\
We would think that \includegraphics{pics/triangle.png} is a \textbf{function} of \includegraphics{pics/circle.png}. The function value in \includegraphics{pics/triangle.png} depends on the value in \includegraphics{pics/circle.png}. \\


The pairs $\{ (3, A), (3, B), (4, C) \}$ do not follow the rule for a function.  \\
3 is in the domain and it is in two different pairs. We would think that \includegraphics{pics/circle.png} is not a function of \includegraphics{pics/triangle.png}.

\end{example}












\end{document}

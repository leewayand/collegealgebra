\documentclass{ximera}

%\usepackage{todonotes}

\newcommand{\todo}{}

\usepackage{esint} % for \oiint
\graphicspath{
  {./}
  {ximeraTutorial/}
}

\newcommand{\mooculus}{\textsf{\textbf{MOOC}\textnormal{\textsf{ULUS}}}}

\usepackage{tkz-euclide}
\tikzset{>=stealth} %% cool arrow head
\tikzset{shorten <>/.style={ shorten >=#1, shorten <=#1 } } %% allows shorter vectors

\usetikzlibrary{backgrounds} %% for boxes around graphs
\usetikzlibrary{shapes,positioning}  %% Clouds and stars
\usetikzlibrary{matrix} %% for matrix
\usepgfplotslibrary{polar} %% for polar plots
\usetkzobj{all}
\usepackage[makeroom]{cancel} %% for strike outs
%\usepackage{mathtools} %% for pretty underbrace % Breaks Ximera
\usepackage{multicol}
\usepackage{pgffor} %% required for integral for loops


%% http://tex.stackexchange.com/questions/66490/drawing-a-tikz-arc-specifying-the-center
%% Draws beach ball 
\tikzset{pics/carc/.style args={#1:#2:#3}{code={\draw[pic actions] (#1:#3) arc(#1:#2:#3);}}}



\usepackage{array}
\setlength{\extrarowheight}{+.1cm}   
\newdimen\digitwidth
\settowidth\digitwidth{9}
\def\divrule#1#2{
\noalign{\moveright#1\digitwidth
\vbox{\hrule width#2\digitwidth}}}





\newcommand{\RR}{\mathbb R}
\newcommand{\R}{\mathbb R}
\newcommand{\N}{\mathbb N}
\newcommand{\Z}{\mathbb Z}

\newcommand{\sagemath}{\textsf{SageMath}}


%\renewcommand{\d}{\,d\!}
\renewcommand{\d}{\mathop{}\!d}
\newcommand{\dd}[2][]{\frac{\d #1}{\d #2}}
\newcommand{\pp}[2][]{\frac{\partial #1}{\partial #2}}
\renewcommand{\l}{\ell}
\newcommand{\ddx}{\frac{d}{\d x}}

\newcommand{\zeroOverZero}{\ensuremath{\boldsymbol{\tfrac{0}{0}}}}
\newcommand{\inftyOverInfty}{\ensuremath{\boldsymbol{\tfrac{\infty}{\infty}}}}
\newcommand{\zeroOverInfty}{\ensuremath{\boldsymbol{\tfrac{0}{\infty}}}}
\newcommand{\zeroTimesInfty}{\ensuremath{\small\boldsymbol{0\cdot \infty}}}
\newcommand{\inftyMinusInfty}{\ensuremath{\small\boldsymbol{\infty - \infty}}}
\newcommand{\oneToInfty}{\ensuremath{\boldsymbol{1^\infty}}}
\newcommand{\zeroToZero}{\ensuremath{\boldsymbol{0^0}}}
\newcommand{\inftyToZero}{\ensuremath{\boldsymbol{\infty^0}}}



\newcommand{\numOverZero}{\ensuremath{\boldsymbol{\tfrac{\#}{0}}}}
\newcommand{\dfn}{\textbf}
%\newcommand{\unit}{\,\mathrm}
\newcommand{\unit}{\mathop{}\!\mathrm}
\newcommand{\eval}[1]{\bigg[ #1 \bigg]}
\newcommand{\seq}[1]{\left( #1 \right)}
\renewcommand{\epsilon}{\varepsilon}
\renewcommand{\phi}{\varphi}


\renewcommand{\iff}{\Leftrightarrow}

\DeclareMathOperator{\arccot}{arccot}
\DeclareMathOperator{\arcsec}{arcsec}
\DeclareMathOperator{\arccsc}{arccsc}
\DeclareMathOperator{\si}{Si}
\DeclareMathOperator{\proj}{\vec{proj}}
\DeclareMathOperator{\scal}{scal}
\DeclareMathOperator{\sign}{sign}


%% \newcommand{\tightoverset}[2]{% for arrow vec
%%   \mathop{#2}\limits^{\vbox to -.5ex{\kern-0.75ex\hbox{$#1$}\vss}}}
\newcommand{\arrowvec}{\overrightarrow}
%\renewcommand{\vec}[1]{\arrowvec{\mathbf{#1}}}
\renewcommand{\vec}{\mathbf}
\newcommand{\veci}{{\boldsymbol{\hat{\imath}}}}
\newcommand{\vecj}{{\boldsymbol{\hat{\jmath}}}}
\newcommand{\veck}{{\boldsymbol{\hat{k}}}}
\newcommand{\vecl}{\boldsymbol{\l}}
\newcommand{\uvec}[1]{\mathbf{\hat{#1}}}
\newcommand{\utan}{\mathbf{\hat{t}}}
\newcommand{\unormal}{\mathbf{\hat{n}}}
\newcommand{\ubinormal}{\mathbf{\hat{b}}}

\newcommand{\dotp}{\bullet}
\newcommand{\cross}{\boldsymbol\times}
\newcommand{\grad}{\boldsymbol\nabla}
\newcommand{\divergence}{\grad\dotp}
\newcommand{\curl}{\grad\cross}
%\DeclareMathOperator{\divergence}{divergence}
%\DeclareMathOperator{\curl}[1]{\grad\cross #1}
\newcommand{\lto}{\mathop{\longrightarrow\,}\limits}

\renewcommand{\bar}{\overline}

\colorlet{textColor}{black} 
\colorlet{background}{white}
\colorlet{penColor}{blue!50!black} % Color of a curve in a plot
\colorlet{penColor2}{red!50!black}% Color of a curve in a plot
\colorlet{penColor3}{red!50!blue} % Color of a curve in a plot
\colorlet{penColor4}{green!50!black} % Color of a curve in a plot
\colorlet{penColor5}{orange!80!black} % Color of a curve in a plot
\colorlet{penColor6}{yellow!70!black} % Color of a curve in a plot
\colorlet{fill1}{penColor!20} % Color of fill in a plot
\colorlet{fill2}{penColor2!20} % Color of fill in a plot
\colorlet{fillp}{fill1} % Color of positive area
\colorlet{filln}{penColor2!20} % Color of negative area
\colorlet{fill3}{penColor3!20} % Fill
\colorlet{fill4}{penColor4!20} % Fill
\colorlet{fill5}{penColor5!20} % Fill
\colorlet{gridColor}{gray!50} % Color of grid in a plot

\newcommand{\surfaceColor}{violet}
\newcommand{\surfaceColorTwo}{redyellow}
\newcommand{\sliceColor}{greenyellow}




\pgfmathdeclarefunction{gauss}{2}{% gives gaussian
  \pgfmathparse{1/(#2*sqrt(2*pi))*exp(-((x-#1)^2)/(2*#2^2))}%
}


%%%%%%%%%%%%%
%% Vectors
%%%%%%%%%%%%%

%% Simple horiz vectors
\renewcommand{\vector}[1]{\left\langle #1\right\rangle}


%% %% Complex Horiz Vectors with angle brackets
%% \makeatletter
%% \renewcommand{\vector}[2][ , ]{\left\langle%
%%   \def\nextitem{\def\nextitem{#1}}%
%%   \@for \el:=#2\do{\nextitem\el}\right\rangle%
%% }
%% \makeatother

%% %% Vertical Vectors
%% \def\vector#1{\begin{bmatrix}\vecListA#1,,\end{bmatrix}}
%% \def\vecListA#1,{\if,#1,\else #1\cr \expandafter \vecListA \fi}

%%%%%%%%%%%%%
%% End of vectors
%%%%%%%%%%%%%

%\newcommand{\fullwidth}{}
%\newcommand{\normalwidth}{}



%% makes a snazzy t-chart for evaluating functions
%\newenvironment{tchart}{\rowcolors{2}{}{background!90!textColor}\array}{\endarray}

%%This is to help with formatting on future title pages.
\newenvironment{sectionOutcomes}{}{} 



%% Flowchart stuff
%\tikzstyle{startstop} = [rectangle, rounded corners, minimum width=3cm, minimum height=1cm,text centered, draw=black]
%\tikzstyle{question} = [rectangle, minimum width=3cm, minimum height=1cm, text centered, draw=black]
%\tikzstyle{decision} = [trapezium, trapezium left angle=70, trapezium right angle=110, minimum width=3cm, minimum height=1cm, text centered, draw=black]
%\tikzstyle{question} = [rectangle, rounded corners, minimum width=3cm, minimum height=1cm,text centered, draw=black]
%\tikzstyle{process} = [rectangle, minimum width=3cm, minimum height=1cm, text centered, draw=black]
%\tikzstyle{decision} = [trapezium, trapezium left angle=70, trapezium right angle=110, minimum width=3cm, minimum height=1cm, text centered, draw=black]


\title{Abstract Thinking}

\begin{document}

\begin{abstract}
%Stuff can go here later if we want!
\end{abstract}

\maketitle

\begin{sectionOutcomes}

After completing this section, students should understand relations and functions as mathematical tools. 

\begin{itemize}
\item Students .
\item Students .
\end{itemize}

\end{sectionOutcomes}




\begin{exercise}
\quad \\
Define the relation Spelling as follows.

The first set is all of the English words in the dictionary. The second set is the English alphabet. An ordered pair (word, letter) is in the relation if letter is a letter in the spelling of word.


\begin{problem} Is the pair (hat, a) in Spelling? \wordChoice{\choice[correct]{Yes} \choice{No}}
\begin{feedback}
The letter \textit{a} is used in the spelling of the word \textit{hat}.
\end{feedback}
\end{problem}


\begin{problem} Is the pair (s, shoe) in Spelling? \wordChoice{\choice[correct]{Yes} \choice{No}}
\begin{feedback}
The letter \textit{s} is used in the spelling of the word \textit{shoe}.
\end{feedback}
\end{problem}


\begin{problem} Is the pair (a, a) in Spelling? \wordChoice{\choice[correct]{Yes} \choice{No}}
\begin{feedback}
The letter \textit{a} is used in the spelling of the word \textit{a}.
\end{feedback}
\end{problem}


\begin{problem} How many pairings contain "hat"? \wordChoice{\choice{0} \choice{1} \choice{2} \choice[correct]{3}}
\begin{feedback}
(h, hat), (a, hat), and (t, hat)
\end{feedback}
\end{problem}


\begin{problem} Is every word in a pairing? \wordChoice{\choice[correct]{Yes} \choice{No}}
\begin{feedback}
Every word has at least one letter in its spelling.
\end{feedback}
\end{problem}


\begin{problem} Is every letter in a pairing? \wordChoice{\choice[correct]{Yes} \choice{No}}
\begin{feedback}
Every letter is in the spelling of at least one word.
\end{feedback}
\end{problem}




\end{exercise}





\begin{exercise}
\quad \\
The Closterman family plays Secret Santa for Christmas. Each person selects anther person?s name and buys them a present. This is a relation. We?ll call it the Secret Santa relation. The two sets are the same set ? the family members. A person is related to a second person, if the first person selected the second person? name. The table below diagrams the connections. The first person is in the top row. The second person is in the bottom row.





\begin{center}
\[
\begin{array}{|c|c|c|c|c|c|}
\hline
Sue & Bart & Jason & Kris & Velma & Winona \\\hline
Bart & Jason & Sue & Winona & Kris & Velma  \\\hline
\end{array}
\]
\end{center}


\begin{problem} Is the pair (Jason, Sue) in Secret Santa? \wordChoice{\choice[correct]{Yes} \choice{No}}
\begin{feedback}
Jason selected Sue's name.
\end{feedback}
\end{problem}


\begin{problem} Is the pair (Kris, Velma) in Secret Santa? \wordChoice{\choice{Yes} \choice[correct]{No}}
\begin{feedback}
Kris did not select Velma's name.
\end{feedback}
\end{problem}


\begin{problem} Is the pair (Bart, Jason) $\in$ Secret Santa? \wordChoice{\choice[correct]{Yes} \choice{No}}
\begin{feedback}
Bart did select Jason's name.
\end{feedback}
\end{problem}


\begin{problem} Is it possible to a pair like (person, person) to be in Secret Santa? \wordChoice{\choice{Yes} \choice[correct]{No}}
\begin{feedback}
You cannot pick yourself for Secret Santa.
\end{feedback}
\end{problem}




\end{exercise}








\begin{exercise}
\quad \\
Next year there will be a new Secret Santa relation for the Closterman family.


\begin{problem} Is it possible that this new relation could be symmetric? \wordChoice{\choice[correct]{Yes} \choice{No}}
\begin{feedback}
There are six people. Pair them off and have them select each other's name.
\end{feedback}
\end{problem}



\begin{problem} Is it possible that this new relation could be symmetric? \wordChoice{\choice{Yes} \choice[correct]{No}}
\begin{feedback}
A person cannot select their own name for Secret Santa.
\end{feedback}
\end{problem}




\begin{problem} Is it possible that this new relation could be transitive? \wordChoice{\choice{Yes} \choice[correct]{No}}
\begin{feedback}
That would mean one person would have selected more than one name.
\end{feedback}
\end{problem}


\end{exercise}









\end{document}

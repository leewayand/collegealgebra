\documentclass{ximera}

\input{../../../preamble.tex}



\outcome{Identify pairs in a relation.}


\begin{document}



 

\begin{exercise}
Consider the \dfn{greatest integer function}.  This function maps
any real number $x$ to the greatest integer less than or equal to $x$.
%\[
%f(x) = \parbox{3in}{The function that maps any real number $x$
%  to the greatest integer less than or equal to $x$.}
%\]
People sometimes write this as $f(x) = \lfloor x\rfloor$, where those
funny symbols mean exactly the words above describing the
function. For your viewing pleasure, here is a graph of the greatest
integer function:
\begin{image}
\begin{tikzpicture}
	\begin{axis}[
            domain=-2:4,
            width=6in,
            height=1in,
            axis lines =middle, xlabel=$ $, ylabel=$ $,
            ymajorticks=false,
            every axis y label/.style={at=(current axis.above origin),anchor=south},
            every axis x label/.style={at=(current axis.right of origin),anchor=west},
            clip=false,
            %axis on top,
          ]
          \addplot [textColor, very thin, domain=(-5:5)] {0}; % puts the axis back, axis on top clobbers our open holes
          \addplot [textColor, very thin] plot coordinates {(0,-0.3) (0,0.3)}; % puts the axis back, axis on top clobbers our open holes

          \addplot [very thick, penColor, domain=(0:1)] {0.1};
          \addplot[color=penColor,fill=penColor,only marks,mark=*] coordinates{(0,0.1)};  %% closed hole          
          \addplot[color=penColor,fill=background,only marks,mark=*] coordinates{(1,0.1)};  %% open hole

        \end{axis}
\end{tikzpicture}
%% \caption{A plot of $f(x)=\lfloor x\rfloor$. Here we can see that for each input (a
%%   value on the $x$-axis), there is exactly one output (a value on the
%%   $y$-axis).}
%% \label{plot:greatest-integer fxn}
\end{image}
Observe that here we have multiple inputs that give
the same output.  This is not a problem! To be a function, we
merely need to check that for each input, there is exactly one output,
and this condition is satisfied.
\end{exercise}











\begin{exercise}
How many items are in the codomain of sparrow? $\answer{5}$
\end{exercise}










\end{document}

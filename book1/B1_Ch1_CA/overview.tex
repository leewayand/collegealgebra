\documentclass{ximera}

%\usepackage{todonotes}

\newcommand{\todo}{}

\usepackage{esint} % for \oiint
\graphicspath{
  {./}
  {ximeraTutorial/}
}

\newcommand{\mooculus}{\textsf{\textbf{MOOC}\textnormal{\textsf{ULUS}}}}

\usepackage{tkz-euclide}
\tikzset{>=stealth} %% cool arrow head
\tikzset{shorten <>/.style={ shorten >=#1, shorten <=#1 } } %% allows shorter vectors

\usetikzlibrary{backgrounds} %% for boxes around graphs
\usetikzlibrary{shapes,positioning}  %% Clouds and stars
\usetikzlibrary{matrix} %% for matrix
\usepgfplotslibrary{polar} %% for polar plots
\usetkzobj{all}
\usepackage[makeroom]{cancel} %% for strike outs
%\usepackage{mathtools} %% for pretty underbrace % Breaks Ximera
\usepackage{multicol}
\usepackage{pgffor} %% required for integral for loops


%% http://tex.stackexchange.com/questions/66490/drawing-a-tikz-arc-specifying-the-center
%% Draws beach ball 
\tikzset{pics/carc/.style args={#1:#2:#3}{code={\draw[pic actions] (#1:#3) arc(#1:#2:#3);}}}



\usepackage{array}
\setlength{\extrarowheight}{+.1cm}   
\newdimen\digitwidth
\settowidth\digitwidth{9}
\def\divrule#1#2{
\noalign{\moveright#1\digitwidth
\vbox{\hrule width#2\digitwidth}}}





\newcommand{\RR}{\mathbb R}
\newcommand{\R}{\mathbb R}
\newcommand{\N}{\mathbb N}
\newcommand{\Z}{\mathbb Z}

\newcommand{\sagemath}{\textsf{SageMath}}


%\renewcommand{\d}{\,d\!}
\renewcommand{\d}{\mathop{}\!d}
\newcommand{\dd}[2][]{\frac{\d #1}{\d #2}}
\newcommand{\pp}[2][]{\frac{\partial #1}{\partial #2}}
\renewcommand{\l}{\ell}
\newcommand{\ddx}{\frac{d}{\d x}}

\newcommand{\zeroOverZero}{\ensuremath{\boldsymbol{\tfrac{0}{0}}}}
\newcommand{\inftyOverInfty}{\ensuremath{\boldsymbol{\tfrac{\infty}{\infty}}}}
\newcommand{\zeroOverInfty}{\ensuremath{\boldsymbol{\tfrac{0}{\infty}}}}
\newcommand{\zeroTimesInfty}{\ensuremath{\small\boldsymbol{0\cdot \infty}}}
\newcommand{\inftyMinusInfty}{\ensuremath{\small\boldsymbol{\infty - \infty}}}
\newcommand{\oneToInfty}{\ensuremath{\boldsymbol{1^\infty}}}
\newcommand{\zeroToZero}{\ensuremath{\boldsymbol{0^0}}}
\newcommand{\inftyToZero}{\ensuremath{\boldsymbol{\infty^0}}}



\newcommand{\numOverZero}{\ensuremath{\boldsymbol{\tfrac{\#}{0}}}}
\newcommand{\dfn}{\textbf}
%\newcommand{\unit}{\,\mathrm}
\newcommand{\unit}{\mathop{}\!\mathrm}
\newcommand{\eval}[1]{\bigg[ #1 \bigg]}
\newcommand{\seq}[1]{\left( #1 \right)}
\renewcommand{\epsilon}{\varepsilon}
\renewcommand{\phi}{\varphi}


\renewcommand{\iff}{\Leftrightarrow}

\DeclareMathOperator{\arccot}{arccot}
\DeclareMathOperator{\arcsec}{arcsec}
\DeclareMathOperator{\arccsc}{arccsc}
\DeclareMathOperator{\si}{Si}
\DeclareMathOperator{\proj}{\vec{proj}}
\DeclareMathOperator{\scal}{scal}
\DeclareMathOperator{\sign}{sign}


%% \newcommand{\tightoverset}[2]{% for arrow vec
%%   \mathop{#2}\limits^{\vbox to -.5ex{\kern-0.75ex\hbox{$#1$}\vss}}}
\newcommand{\arrowvec}{\overrightarrow}
%\renewcommand{\vec}[1]{\arrowvec{\mathbf{#1}}}
\renewcommand{\vec}{\mathbf}
\newcommand{\veci}{{\boldsymbol{\hat{\imath}}}}
\newcommand{\vecj}{{\boldsymbol{\hat{\jmath}}}}
\newcommand{\veck}{{\boldsymbol{\hat{k}}}}
\newcommand{\vecl}{\boldsymbol{\l}}
\newcommand{\uvec}[1]{\mathbf{\hat{#1}}}
\newcommand{\utan}{\mathbf{\hat{t}}}
\newcommand{\unormal}{\mathbf{\hat{n}}}
\newcommand{\ubinormal}{\mathbf{\hat{b}}}

\newcommand{\dotp}{\bullet}
\newcommand{\cross}{\boldsymbol\times}
\newcommand{\grad}{\boldsymbol\nabla}
\newcommand{\divergence}{\grad\dotp}
\newcommand{\curl}{\grad\cross}
%\DeclareMathOperator{\divergence}{divergence}
%\DeclareMathOperator{\curl}[1]{\grad\cross #1}
\newcommand{\lto}{\mathop{\longrightarrow\,}\limits}

\renewcommand{\bar}{\overline}

\colorlet{textColor}{black} 
\colorlet{background}{white}
\colorlet{penColor}{blue!50!black} % Color of a curve in a plot
\colorlet{penColor2}{red!50!black}% Color of a curve in a plot
\colorlet{penColor3}{red!50!blue} % Color of a curve in a plot
\colorlet{penColor4}{green!50!black} % Color of a curve in a plot
\colorlet{penColor5}{orange!80!black} % Color of a curve in a plot
\colorlet{penColor6}{yellow!70!black} % Color of a curve in a plot
\colorlet{fill1}{penColor!20} % Color of fill in a plot
\colorlet{fill2}{penColor2!20} % Color of fill in a plot
\colorlet{fillp}{fill1} % Color of positive area
\colorlet{filln}{penColor2!20} % Color of negative area
\colorlet{fill3}{penColor3!20} % Fill
\colorlet{fill4}{penColor4!20} % Fill
\colorlet{fill5}{penColor5!20} % Fill
\colorlet{gridColor}{gray!50} % Color of grid in a plot

\newcommand{\surfaceColor}{violet}
\newcommand{\surfaceColorTwo}{redyellow}
\newcommand{\sliceColor}{greenyellow}




\pgfmathdeclarefunction{gauss}{2}{% gives gaussian
  \pgfmathparse{1/(#2*sqrt(2*pi))*exp(-((x-#1)^2)/(2*#2^2))}%
}


%%%%%%%%%%%%%
%% Vectors
%%%%%%%%%%%%%

%% Simple horiz vectors
\renewcommand{\vector}[1]{\left\langle #1\right\rangle}


%% %% Complex Horiz Vectors with angle brackets
%% \makeatletter
%% \renewcommand{\vector}[2][ , ]{\left\langle%
%%   \def\nextitem{\def\nextitem{#1}}%
%%   \@for \el:=#2\do{\nextitem\el}\right\rangle%
%% }
%% \makeatother

%% %% Vertical Vectors
%% \def\vector#1{\begin{bmatrix}\vecListA#1,,\end{bmatrix}}
%% \def\vecListA#1,{\if,#1,\else #1\cr \expandafter \vecListA \fi}

%%%%%%%%%%%%%
%% End of vectors
%%%%%%%%%%%%%

%\newcommand{\fullwidth}{}
%\newcommand{\normalwidth}{}



%% makes a snazzy t-chart for evaluating functions
%\newenvironment{tchart}{\rowcolors{2}{}{background!90!textColor}\array}{\endarray}

%%This is to help with formatting on future title pages.
\newenvironment{sectionOutcomes}{}{} 



%% Flowchart stuff
%\tikzstyle{startstop} = [rectangle, rounded corners, minimum width=3cm, minimum height=1cm,text centered, draw=black]
%\tikzstyle{question} = [rectangle, minimum width=3cm, minimum height=1cm, text centered, draw=black]
%\tikzstyle{decision} = [trapezium, trapezium left angle=70, trapezium right angle=110, minimum width=3cm, minimum height=1cm, text centered, draw=black]
%\tikzstyle{question} = [rectangle, rounded corners, minimum width=3cm, minimum height=1cm,text centered, draw=black]
%\tikzstyle{process} = [rectangle, minimum width=3cm, minimum height=1cm, text centered, draw=black]
%\tikzstyle{decision} = [trapezium, trapezium left angle=70, trapezium right angle=110, minimum width=3cm, minimum height=1cm, text centered, draw=black]


\title{College Algebra}

\begin{document}

\begin{abstract}
%Stuff can go here later if we want!
\end{abstract}

\maketitle

What is College Algebra? and Why are you studying it?

College Algebra is the first college mathematics course in the STEM (Science, Technology, Engineering, and Mathematics) pathway. It serves as a transition course. You are transitioning from thinking about individual objects to thinking about systems and relations.

\textbf{Numbers to Functions} \\
In previous mathematics courses, students generally encounter a computational view of mathematics in which individual numbers are the objects under investigation. College Algebra introduces a transition from numbers as the main focus of study to functions as the main focus. This transition is vital for success in the STEM pathway.

Almost anyone entering college would agree that numbers are the point of mathematics.  Students have spent years calculating with numbers, tracking down numbers via equations, and cataloging associated procedural skills with algebraic symbols, which represent numbers. Well, that was phase 1.  Phase 1 is now complete.  Phase 2 is about to begin.

The world is too complicated to describe with individual numbers. Even collections of isolated numbers are not enough. We need to understand how collections of numbers relate to each other. These relations are the focus of College Algebra.

Our goal is to create collections of numbers derived from measurements and describe how the collections compare to each other. To do this, we'll create a new tool called a \textbf{function}.

Functions are the main focus of the STEM pathway.  College Algebra and Trigonometry lay the foundation for the Elementary functions and then Calculus develops the analysis of functions.  Functions describe the relationship between collections of numbers and Calculus deepens and expands the analysis of this relationship.

\begin{idea}
In previous algebra courses, you were asked to solve equations.
\[
Solve (x+1)(x-2)(x-3) = 0
\]

Your response was to identify the numbers $-1$, $2$, and $3$. Three individual numbers. But, there is a lot of information lost in this situation.

Instead, we might just collect \textbf{ALL} of the values of $(x+1)(x-2)(x-3)$.  Substitute every real number in for $x$, one at a time, and record all of the values. We would have the three numbers that produce $0$ and everything else.

Identifying the solution set $\{ -1, 2, 3 \}$ is algebraic thinking, which we still want and want to improve.  Thinking of the collection of \textbf{ALL} possible values is function thinking.  This is the type of thinking we will introduce in this course.  This is the type of thinking that calculus expands and deepens.

\end{idea}

There is more to the College Algebra transition.

\textbf{Conceptual Viewpoint} \\
STEM students also need to transition from a procedural viewpoint of mathematics to a conceptual view. In previous mathematics courses, students worked with individual numbers. Solutions consisted of isolated numbers. Working with functions certainly involves individual and isolated numbers. But, these numbers are usually marking change in function behavior. It is the behavior of functions that we want to map out. Rather than a goal of identifying individual numbers, we will use these numbers to partition the collections of measurements, compare the pieces, and categorize the connections. Our viewpoint needs to expand from items to relationships between collections of items.

Calculus will then expand this idea.  The collections of items will be collections of functions and then there will be new functions created that describe relationships between the collections of functions.

There is more to the College Algebra transition.







\textbf{Algebra vs. Analysis} \\
Mathematics is a huge field of study, but from the viewpoint of College Algebra, mathematics is more-or-less split into two pieces: Algebra and Analysis.

Algebra is interested in exactness.  Equals dominates the investigation. Analysis is interested in closeness.  Obtaining equality is not always possible.  Every situation has exact measurements. These are often out of reach.  Then the investigation turns to approximating these measurements.

For the most part, College Algebra will investigate equality.  We will want to know exactly what is happening. Calculus will pick up the analysis story. College Algebra will also introduce some of the beginning concepts of analysis and begin the next transition to Calculus.
There is more to the College Algebra transition.

\textbf{What? to How?} \\
Since previous mathematics courses focused on procedures, the solutions have generally be individual numbers.  ?What? has been the answer.  That is about to change. College mathematics courses generally want to know ?How?. Solutions now consist of explanations. 
The mathematical situations will soon become too complicated to hold in your head.  So, there is plenty of skepticism about conclusions. It isn?t enough to identify an answer, people want to know how you came up with the answer.  That is how they are going to judge their confidence in your answer.  

There is more to the College Algebra transition.

\textbf{Infinity} \\
The best thing about the transition to college mathematics is infinity finally makes an appearance. Our collections of numbers will include an infinite amount of numbers. Our functions will compare infinite collections of numbers. Our analysis will partition and categorize an infinite number of pieces of an infinite collection of numbers. Learning how to adjust your thinking to infinite sets is a major step in college mathematics.


All of this fits together in a coherent story. But, it will take a while to understand the story.  College Algebra is just the first step. Your approach will have to change from asking to be shown how to do procedures to asking people to respond to your thoughts. This book will give you plenty of opportunities to develop your own thoughts, your own approach, your own understanding of mathematics.

\textbf{Start Thinking} \\
If you are ready to study College Algebra, then you have taken many mathematics courses over many years. You have significant experience with many mathematical objects and concepts. You have a parallel experience with Language Arts.  You have taken many English classes and have significant experience with many language arts objects and concepts. These subjects are not as far apart as you might think.

Think of the following questions. But, don't give examples. Don't say "like". Think up an explanation that describes the object or concept.  
\begin{itemize}
\item Don't give an example as your answer.
\item Don't say "like", which is a prelude to giving an example.
\item An explanation is not an example or an instance or an analogy.
\end{itemize}
Give an explanation or description.

\begin{enumerate}
\item What is a number? What is the purpose of a number?
\item What is a function? What is the purpose of a function?
\item What is a noun? What is the purpose of a noun?
\item What is a variable? What is the purpose of a variable?
\item What is a verb? What is the purpose of a verb?
\item What is an equation? What is the purpose of an equation?
\item What is a sentence? What is the purpose of a sentence?
\item What is the variable in this equation:  $3x+4=19$ ? 
\item What is a question mark? What is the purpose of a question mark?
\item What is a fraction?
\item What is a sum of two numbers? What does it mean to add two numbers together? 
\item What is a sum of two functions? What does it mean to add two functions together?
\item What is the graph of a function? What is the graph of an equation? What is the difference?
\item What is a pattern? How do you identify patterns?
\item How do you describe patterns you notice to other people?
\item What is a paragraph? What is the purpose of a paragraph?
\item $\sqrt{2}$ represents the square root of two. What exactly is the square root of two?
\item What is the difference between exact and approximate?
\item What is infinity? What does the word "all" mean to you?
\item What is an expression?  What is a formula? What is the domain of a function?
\item What is the difference between clear and confusing? 
\item What does "simplify" mean?
\item What is the difference between an answer and an explanation? 
\item What is an answer? What is a solution?
\item How many types of representations can you create for the number four? Do it.
\end{enumerate}



Those thoughts are the thoughts of College Algebra.?






\end{document}

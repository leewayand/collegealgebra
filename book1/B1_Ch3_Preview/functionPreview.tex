\documentclass{ximera}

\input{../../preamble.tex}

\title{Function Preview}

\begin{document}

\begin{abstract}
%Stuff can go here later if we want!
\end{abstract}

\maketitle





\begin{sectionOutcomes}

Functions are packages. \\
The package contains three sets:

\begin{itemize}
\item a Domain
\item a CoDomain
\item a set of Pairs
\end{itemize}

The package follows one rule:
\begin{itemize}
\item Every domain item is in EXACTLY one pair.
\end{itemize}

\end{sectionOutcomes}




\quad \\


Relations allow too many connections for us.  We will investigate a simpler type of relation. Functions are relations with one property, which will significantly simplify the situation.

Functions are relations in which items in the first set (the domain) are connected to exactly one item in the second set (codomain).  This will clear up the picture and allow us to map a logical structure of various types of functions.

First we need a generic foundation for functions that we can later fill in with specific information for any situation we encounter. 

Our generic viewpoint sees functions as packages holding three sets:
\begin{itemize}
\item Domain 
\item Codomain 
\item Pairs
\end{itemize}

\quad \\

Along with one rule: each domain item has to be included in exactly one pair. (exactly one!)

Exactly one means exactly one.  Not two. Not zero. Each domain item has a codomain partner and only one codomain partner.

Once we become familiar with this generic viewpoint, we will begin to apply it to numeric situations.  First, we?ll use the rule to decipher pictures and graphs for numeric functions and then we?ll use it to operate formulas for numeric functions.

After we have introduced these tools, we will slowly build our library of the Elementary Functions and discover their indivdual properties and relationships.

But, first, the generic or abstract foundation of functions.

















\end{document}
